\chapter{Introduction}
\section{Air pollution}
Air pollution can be defined as a group of chemicals present in the atmosphere harmful for humans, animals or vegetation. They are mainly caused by human activities, such as transport, industry, or agriculture. But they can also be influenced by other natural sources. Understanding air pollution is important because many health consequences are brought from high pollution levels. One historic event which caused huge consequences is known as the the great smog of 1952. Thousands of persons died in Greater London due to  breathing for various days a highly contaminated atmosphere and many others became ill or experienced retarded symptoms[]. The fog originated from coil burning, vehicle exhaust and other atmospheric factors. Although many human activities introducing pollution particles have changed since then, it became evident the immediate and retarded health impact of pollution particles. More?

Pollution chemicals can be categorized into gaseous pollutants, persistent organic pollutants, heavy metals and particulate matter. They change on chemical composition, emission sources and impact on health. Gaseous pollutants are sulfur dioxide (SO\textsubscript{2}), nitrogen oxides (NO\textsubscript{x}), carbon monoxide (CO), ozone (O\textsubscript{3}) and volatile organic compounds. The principal source of this gas pollutants is combustion. Nitrogen oxides (NO\textsubscript{x}) is a general term that includes nitric oxide (NO) and nitrogen oxide (NO\textsubscript{2}). NO and NO\textsubscript{2} come from combustion of fossil fuels such as coal and natural gas. The health effects of nitrogen oxides are irritating the lungs and increasing susceptibility to respiratory infections like influenza[].

Air quality is also affected by pollution mixture in further complex chemical structures and by temperature and humidity conditions . NO\textsubscript{2}, PM and O\textsubscript{2}  pollutants get transformed by atmospheric processes making complex to evaluate their individual impact. As an example, ground level ozone is produced when sunlight interacts with NO\textsubscript{2} and volatile organic compounds. Furthermore, NO\textsubscript{2} and other nitrogen oxides also contribute to Particle Matter (PM) generation, making NO\textsubscript{x} a particularly concerning pollutant.

Particulate matter (PM) is a mixture of solid and liquid particles including sulfate, nitrates, ammonia, sodium chloride, black carbon, mineral dust and water. The sources of PM are mainly vehicle, industrial and domestic exhaust, but it is also composed of fires and cigarette smoke. PM are categorized according their diameter size measured in microns (μm, 1×10−6 of a metre), particles smaller than 10 microns are known as PM\textsubscript{10}, also PM\textsubscript{2.5} and PM\textsubscript{1}, smaller than 2.5 microns and 1 micron respectively, are frequently monitored.