\section{Evaluation}
Considering that a deliverable of my project dissertation is a mobile application that includes technical and design aspects, the evaluation should be qualitative and quantitative. It should be quantitative to get measurable indicators of how well the user-requirements were met from a software engineering perspective, and qualitative to get an insight from an interaction-design perspective to collect thoughts and reflections about the product.

\subsection{Technical evaluation}
A technical evaluation will ensure that the software behaves as expected during deployment. This will be done by inspecting and debugging the code through the entire development process. Code inspections are performed by visually inspecting the program coded statements and allow to discover logic and coding errors to spend less time and effort debugging \cite{Myers2011}. The coding errors that are not discovered on the inspection process will show up when executing the program and may be fixed by debugging. Debugging allows to find and correct suspected errors within the program in a effort-less manner by using debugging tools integrated into modern development editors. Once the software development has been exhausted automated tests will be used to discover errors that still may be present.

\subsection{Evaluation with users}
Even though the application will be designed to be understood by everyone, is important for the user testers to be able to handle computers and smartphones in a confident way. Otherwise, they will not be able to complete the tasks within the system independently and may add bias to the gathered results. 

It is of particular interest to gather user levels of satisfactions, and measurable indicators of how well the non-functional and functional requirements were met by the application, in other words, to assess how well people can understand and use a product or prototype.

\quotes{This method is accomplished by identifying representative users, representative tasks, and developing a procedure for capturing the problems that users have in trying to apply a particular software product in accomplishing these tasks.} \cite{Scholtz2003} The advantage is the involvement of the user; however, the test attendants should be a representative group of the potential final users of the product. The selected tasks should aim to be as realistic as possible because \quotes{results are based on actually seeing what aspects of the user interface cause problems for representative users}. \cite{Scholtz2003}

\subsection{Design critique}
The objective of including this method is to gather impressions and thoughts about the system and leave open space to discover what future releases should aim for. According to Blevis et al. \cite{Blevis2007} it allows to understand the development from the perspective of the user. In this case, will try to understand how the users would include such development in their daily lives. For instance, if they would be willing to use it on a regular basis, or if they would use it because its attractive and fun to use, or just because they feel it would benefit their health in some way. All of these would give a wider understanding of the flaws and strengths of the system.

\iffalse
\quotes{Process of discourse on many levels of the nature and effects of an ultimate particular design}. \quotes{Comment on the qualities of an ultimate particular from an holistic perspective, including reason, ethics, and aesthetics as well as minute details of form and external effects on culture}.\cite{Blevis2007}
\fi

