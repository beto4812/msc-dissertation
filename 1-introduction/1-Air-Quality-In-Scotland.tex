In recent years air quality has taken on an increasingly important role in people’s lives. Furthering the case, air pollution has also been recognised by governments as a significant threat to human health. In Scotland alone there are in total 32 ``pollution zones'' that are considered to be in breach of European safety standards \cite{Foe-scotland.org.uk} \cite{OKScotland2015}. The negative impact of poor air on our health, diminishing our life quality and causing cardiovascular and respiratory diseases is of greater concern. As such, new air quality monitoring sensors are being developed and deployed. Some of them are in fixed locations, whilst others are carried around. Data is becoming more available and fine-grained providing a further capability to enable educated decisions.

Through this new outsourced data, it is possible to tackle the pollution problem by taking on a  more personalised approach as ``everyone has different needs and lifestyles that expose them differently to pollution'' \cite{Vazquez2016}. Different population groups are more vulnerable than others. For example, children are more affected than adults because their lungs are still developing. Furthermore, people with respiratory diseases suffer from aggravated symptoms because their airways get irritated with spikes in air pollution. The way data is presented and disseminated to the public is crucial part of the solution. Therefore, digitally enabled visualisation has the potential to unleash the full potential of data as new technologies extract, filter, display and animate data, they evoke the interest and attention of people while providing unique experiences and mechanisms to improve their lives.

\iffalse
The effects of air pollution on human health are still complex to understand and there is much research ongoing on the combination short and long term effects upon a person's health. 
\fi