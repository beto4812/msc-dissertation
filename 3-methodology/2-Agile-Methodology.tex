\section{Agile methodology}
The development process followed an agile methodology as a good software engineering practice. The benefit of this approach is that more focus is paid to the product than to the management process. This has proven to be a strong advantage when working on constantly-changing and time-constrained projects such as this dissertation project.
The agile manifesto \cite{Martin2002} establishes four values to guide a development process: 
\begin{displayquote}
\begin{itemize}
  \item Individuals and interactions over processes and tools
  \item Working software over comprehensive documentation
  \item Customer collaboration over contract negotiation
  \item Responding to change over following a plan. 
\end{itemize}
\end{displayquote}

Agility moves the interest in software development towards product functionality rather than product documentation. It recognises that much time can be spent on documenting and planning and stimulates smart resource usage instead by working closely with the stakeholders, responding to change and prioritising software components that have direct interaction with the user. This allows the developer to output iterations to the user for evaluating earlier in the development process.

\iffalse
\subsection{Self-management}
Versioning software was used to manage the workflow of the different components of the system, the repositories are available online in a public GitHub repository. Links
\fi