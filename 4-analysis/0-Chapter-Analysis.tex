\chapter{Analysis}
The focus of this chapter is to establish and discuss what is the software going to do, and how is going to achieve its intended goals. The actors in defining this are the stakeholders. The process of doing so is through the requirements elicitation. 

\section{Users and stakeholders}
As mentioned before, prior to the start of the development phase, it is important to establish who are the intended users of the application, as well as the stakeholders. 

The following users were depicted for this development:

\begin{itemize}
	\item General public: The general public who is concerned on the air quality on their environment.
    \item Pollution-sensitive public: People at risk of suffering health consequences due to air pollution.
\end{itemize}

In order to gain input from different perspectives, the following stakeholders were involved through the development: 

\begin{itemize}



	\item Researchers 
    \begin{itemize}
      \item Asthma UK researchers: Meetings were arranged with various researchers of the asthma UK group. They were involved early in the design of a first prototype, by providing ideas and highlighting issues that may arise from a medical researcher perspective. The queried researches were:  Dr. Aziz Sheikh, director of the research group. Dr. Soyiri Ireneous, and Dr. Lynn Morrice.

      \item University of Edinburgh researchers: My supervisor and co-supervisor

	\end{itemize}


%    The following issues were raised in the meetings:
%    \begin{itemize}
%		\item Applications that rely on user input are not sustainable.
%		\item Review past approaches/work to avoid known problems.
%		\item An application should give actionable feedback rather than just displaying %information.
%        \item Ask the users to know what they want.
%	\end{itemize}
    
	\item Domain experts    
   \begin{itemize}
      \item Chief Technology Officer: Simon Chapple acts as chief technology officer of Datalytics technology, a company who has expertise on mobile applications and internet of things. He was asked for feedback about the design of a prototype of the application.
	\end{itemize}
    
%    The following recommendations were raised in the meeting:
%    \begin{itemize}
%		\item User interfaces should aim to be self-explanatory. 
%		\item Colors should be consistent and provide meaning, as users associate colors to %real world entities.
%		\item Further consideration should be taken when choosing a native vs hybrid %development approach.
%		\item Welcome screens should have an immediate, understandable objective.   
%	\end{itemize}
%"I wonder how you did the animations behave that way"

	\item Users as stakeholders: The final users were involved in three stages, in the requirements elicitation, as co-designers for the prototypes, and as evaluator of the final product. 
    \begin{itemize}

		    \item The Asthma UK patient group: The patient group of the Asthma UK for applied research was involved in two phases: The requirements elicitation via an online questionnaire, and the evaluation of the final product.
			\item General public: They were involved in three phases: The requirements elicitation via an online questionnaire, the evaluation of  prototypes and the evaluation of the final product. 
			\item How to ref this?: Mic starbuck, an activist who is involved in the guverment council and suffers from asthma was involved in the very first ideas of the design of the application, from his input, we were able to understand in a broad sense, what might an asthma sufferer need. He was therefore involved in the requirements elicitation.  


	\end{itemize}

\end{itemize}

\section{Requirements elicitation}
Functional requirements are and non functional requirements are.... 
The main functional-requirements were elicited from the users because they were the first reason of the development. The way of doing it was through interviews and questionnaires.

%Feedback from researchers and domain experts was taken into account as non-functional %requirements, they did not ask for any function per se. but just established guidelines a %


\subsection{Interview}



Mic. Starbuck is a valuable input because he is a targeted final user, as asthmasufferer he pointed out the following needs: 

\subsection{Questionnaire}

A questionnaire was constructed taking into account the age of the u


\subsection{Functional requirements}



\subsection{Non functional requirements}

\begin{itemize}
	\item Learnability
    \item 
\end{itemize}

The user stories that translated from requirements
\section{System architecture}
A tiered architecture was chosen to meet the requirements. There are three tiers, the android application is the presentation tier,
the logic tier runs on Amazon EC25, and the database tier runs on Amazon dynamo6. This approach allows modularizing the com- ponents of the system, giving the Android device the rendering tasks to improve performance and executing resource-consuming tasks under an amazon virtual server.

\section{Technological choices}
\subsection{Platform choice}
\subsection{Air quality data source}
\subsection{Schema-free database}
\subsection{Infrastructure as a service}
\subsection{Amazon web services SDK}
\section{Android specifities}
\subsection{Material design}
\subsection{Fragments}