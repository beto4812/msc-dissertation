In the last few years, air quality is becoming increasingly important in everyone's daily life as people and government institutions are becoming more aware of the hazards of air pollution. In Scotland alone there are in total 32 ``pollution zones'' which are considered to break European safety standards \cite{Foe-scotland.org.uk}. Also, as far as humans are concerned air pollution has a negative impact towards our health, diminishing our life quality and causing cardiovascular and respiratory diseases. Because of this, new air quality monitoring sensors are being developed and deployed. Some of them are in fixed locations, but some others are carried around by people trough their activities. Data is becoming more available and fine-grained providing a further capability to enable educated decisions.

Through this new outsourced data, it is possible to tackle the pollution problem by taking a  more personalised approach as ``everyone has different needs and lifestyles that expose them differently to pollution'' \cite{Vazquez2016}. For instance, it is known that some population groups are more vulnerable than others, for example, children are more affected than adults because their lungs are still developing. Furthermore, people with respiratory diseases suffer from aggravated symptoms because their airways get irritated when air pollution spikes are triggered. The way data is presented and disseminated to the public takes a crucial part on the problem solution. Therefore, digitally enabled visualisations are an excellent tool that unleashes the full potential of data, as new technologies are available to extract, filter, display and animate it evoking the interest and attention of the people while providing unique experiences and mechanisms to improve their lives.

\iffalse
The effects of air pollution on human health are still complex to understand and there is much research ongoing on the combination short and long term effects upon a person's health. 
\fi