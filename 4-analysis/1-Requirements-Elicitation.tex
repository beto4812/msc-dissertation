\section{Requirements elicitation}
For the requirements elicitation, I have had meetings with different persons interested in the project. The first person who showed
interest was Mic Starbuck, who provided many requirements from the point of view of a person suffering from asthma that served as a starting point development process. Another person who gave a research point of view was Dr. Soyiri Ireneous, who’s research is on asthma and gave the criticism that a support decision application, should provide an actionable advice for persons to follow up before a contingency arise, and not during the contingency. Also, I had a meeting with Simon Chapple, CTO of Datalytics Technology, who gave a point of view related to the human-computer interaction issues that should be considered in the application. And finally, the end-users are being queried through a survey available online1. One survey was sent to the Asthma UK Centre for Applied Research2 asthmatic advisory group, and the other to the general public through the Informatics Facebook group3. At the moment I have gathered ten responses from well sampled group ages. One challenge that arises at this point is that requirements from different stakeholders don’t completely agree, or are not feasible given the short period of time of the project. The proposed solution is to agree on requirements that are feasible and that could add the most value to the final product.
