\section{Air quality visualizations}
The problem of making the air quality problem visible to the general population has been addressed before, from map-based representations to wearable devices and in-city displays. These approaches contemplate new or known ways to tackle the problem of air quality, and therefore; making the invisible visible.

\subsection{Table based}
The most basic representation of air quality data is as tabular data, generally including some colours to indicate the quality level of the measurement as shown in figure \ref{fig:table_based_visualization}. From these kinds of representations is easy to read information from multiple places. One drawback is that much information is available for the reader to process easily, not always pertinent to the reader current location or requirements. Another issue is that the user requires a previous understanding of the terms and measurements used to annotate the table like \quotes{hourly mean} or \quotes{runing 8 hour mean}. 

\begin{figure}[H]
\begin{adjustbox}{width=1\textwidth,center=\textwidth}
  \centering
  \includegraphics[scale=1]{images/tabular_data.png}
\end{adjustbox}
  \caption[Tabular visualization]{Tabular visualization \cite{DepartmentforEnvironment}}
  \label{fig:table_based_visualization}
\end{figure}

\subsection{Map based}
Through maps people can understand where measurements are being made, they are also good to discover pollution-safe routes when walking around the city or to find out which spots in the city are well suited to realise physical activity.  Some representations include colour indicators to show how good or bad the measurements and can be enabled to discover history data by selecting previous dates. However; it is unlikely that one is interested in the measurements for all Scotland's regions, and it is hard to visualise correctly overlaid numeric data on small screens.

\begin{figure}[H]
\begin{adjustbox}{width=1\textwidth,center=\textwidth}
  \centering
  \includegraphics[scale=.30]{images/map_visualization.png}
\end{adjustbox}
  \caption[Map-based visualization]{Map-based visualization \cite{Scottishairquality.co.uk2016}}
  \label{fig:web_based_desktop_visualization}
\end{figure}


\subsection{Line based}
Line graphs serve well to represent variations of data in defined periods of time allowing to depict patterns and predictions within the data quickly. It is also possible to include various datasets in one single graph, enabling comparison between them. The inAir project \cite{Kim2013} utilises a line graph to represent indoor particle matter count over time. Their findings suggested that this approach allows users to reflect on their behaviours and air quality status, persuading them to modify their practices to decrease their exposure to air pollution. 

\begin{figure}[H]
\begin{adjustbox}{width=1\textwidth,center=\textwidth}
  \centering
  \includegraphics[scale=1]{images/InAir.png}
\end{adjustbox}
  \caption[inAir project: line-based visualizations]{inAir: line-based visualization\cite{Kim2013}.}
  \label{fig:line_based_inAir}
\end{figure}


\subsection{Photo based}
Photo-based representations as proposed by Lin \cite{Lin2014} allow people to understand levels of pollution by taking pictures of the current environment. The pictures are later adjusted using a filter that represents the air quality status (NO2) at that location. This serves as a playful way to visualise air quality while performing a habitual activity as it is taking pictures. On the other hand, this project only aggregates nitrogen dioxide readings, excluding other pollutants that may be pertinent, as well as not going further to indicate what should a person do given the pollution levels.

\begin{figure}[H]
\begin{adjustbox}{width=.8\textwidth,center=\textwidth}
  \centering
  \includegraphics[scale=.4]{images/instaNO2.jpg}
\end{adjustbox}
  \caption[InstaNO2 project: photo-based visualizations]{InstaNO2: photo-based visualizations \cite{Lin2014}.}
  \label{fig:photo_based_instaNO2}
\end{figure}

\subsection{Tangible approaches}
Other approaches have made use of tangible objects to visualise air pollution in a more engaging way. The Human sensor project \cite{InvisibleDust2016} employs wearable art and in-city performances to reveal changes in pollution. Similarly, WearAir, \cite{Kim2010} is an expressive T-shirt that senses VOCs and expresses the readings making use of embedded lights. Kuznetsov et al. \cite{Kuznetsov2011} developed glowing balloons with air quality sensors for users to explore urban air quality. The \textit{IBM think} exhibit \cite{IBM2012} is a 123-foot long digital wall display located in New York city which renders art patterns and real time streaming visualisations showing air quality data as well as traffic flow, energy, and water usage. This is intended to be an immersive in-city experience for users to explore and understand the role of data in the world. Tangible approaches are useful because the evoke curiosity and surprise while gaining an insight of what is happening. In contrast, they are not intended for personal day to day usage but as a way of expression and art. 

\begin{figure}[H]
\begin{adjustbox}{width=.8\textwidth,center=\textwidth}
  \centering
  \includegraphics[scale=.4]{images/think_human_sensor_balloons.jpg}
\end{adjustbox}
  \caption[Tangible visualizations]{The human-sensor on the top-left \cite{InvisibleDust2016}, glowing balloons on the top-right \cite{Kuznetsov2011}, and IBM think on the bottom \cite{IBM2012}.}
  \label{fig:photo_based_instaNO2}
\end{figure}

