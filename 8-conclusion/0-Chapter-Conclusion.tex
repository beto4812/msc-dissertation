\chapter{Conclusions and Future Work}
This chapter builds on the main objectives of my dissertation, stating what has been accomplished so far and  looking at it from an integrative perspective in order to discover what would be interesting to research further. 

\section{Conclusions}

Approaches that tackle the pollution problem through visualisation are still evolving as they depend on a number of factors to converge: the source and quality of the data employed, the vehicle for visualising the data in an attractive and engaging way and the way the user is taken into account in their particular context. Much research has been done in the individual disciplines, but integrating them towards a valuable tool is still a complex task.

The contribution in regard the problem required an interdisciplinary path. A literature review concerning similar proposals has been delivered to understand the underlying issues and needs. Furthermore, research has been conducted to view the problem from the perspective of the users. All of this allowed for the development of three iterations of a visualisation tool in the form of an app.

The delivered application addressed technical but also design issues. It required the development of a bot to automate the delivery of pollution data, as well as an expert system to translate health advice according to the COMEAP air quality report. The design issues addressed the visualisations per se, which enabled the users to see beyond the raw data to enable decision-making in new personal ways. 

\section{Future work}
Regarding what should be interesting for further research, much care was taken on this project to gather the most accurate data available. However, having several fixed sensors in the city is not enough. This is because the personal exposure to pollutants varies from one place to another, potentially on a street level resolution. What should be more adequate is the use of small cheap portable sensors able to measure a wide range of pollutants and other atmospheric variables. Furthermore, each sensor should be connected to a network in a similar way to the AURN network, being able to collect and aggregate data from multiple sensors on-line  exposed through a public API to feed many kinds of applications and other software.

Another interesting line of work relies on the \textit{tracking} part that was conceived at the beginning of the project. This by providing accessible interfaces to make tracking less cumbersome and tedious. When the sensitive participants were asked if they were already using means to keep track of their symptoms, many answered positively. Furthermore, some responded they would be willing to input symptoms data into the app indicating that there is an opportunity on integrating symptoms data and air quality data. This would enable valuable recommendations based on the correlations between individual pollutant spikes and symptoms at different places. For instance, the system could learn what are the most harmful pollutants for the user and be warned when they exceed a certain threshold.

Lastly, the visualisations can be optimised to include more functionalities such as an extended map to visualise the impact of the user while walking through the city, in order to discover healthier places to carry out activities. Currently, the app is not tracking any data from the user, but if this was the case and if a reliable personal sensor were available, the app could offer an automatic log of the user locations and the impact of each place. 