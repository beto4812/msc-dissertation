Air pollution can be defined as a group of chemicals present in the atmosphere harmful for humans, animals or vegetation. They are mainly caused by human activities, such as transport, industry, or agriculture. But they can also be influenced by other natural sources. Understanding air pollution is important because many health consequences are brought from high pollution levels. 
One historic event which caused huge consequences is known as the the great smog of 1952. Thousands of persons died in Greater London due to  breathing for various days a highly contaminated atmosphere and many others became ill or experienced retarded symptoms[]. The fog originated from coil burning, vehicle exhaust and other atmospheric factors. Although many human activities introducing pollution particles have changed since then, it became evident the immediate and retarded health impact of pollution particles. More?
Pollution chemicals can be categorized into gaseous pollutants, persistent organic pollutants, heavy metals and particulate matter. They change on chemical composition, emission sources and impact on health. 
Gaseous pollutants are sulfur dioxide (SO2), nitrogen oxides (NOx), carbon monoxide (CO), ozone (O3) and volatile organic compounds. The principal source of this kind of pollutants is combustion.
Nitrogen oxides (NOx) is a general term that includes nitric oxide (NO) and nitrogen oxide (NO2). NO and NO2 come from combustion of fossil fuels such as coal and natural gas. The health effects of nitrogen oxides are irritating the lungs and increasing susceptibility to respiratory infections like influenza[].
Increase susceptibility to respiratory infections. 
