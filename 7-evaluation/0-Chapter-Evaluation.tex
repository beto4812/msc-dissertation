\chapter{Evaluation}
At this point, the development of the final product was extensively carried out with different design and development iterations with feedback received from the stakeholders. The web-crawler component of the back-end system has been already working for around one month; and many senor readings are available for the users to test with the real gathered data. However, because the air-quality levels in Scotland are always generally good; there was no space to showcase the full capabilities of the system, i.e, under very polluted environments. Because of this, a small 'demo mode' was added to the system to give the opportunity to experiment it under very polluted scenarios. 

\subsection{User sampling}
According to Scholtz \cite{Scholtz2003}, the evaluation should be realized with around 5 participants per representative class of users. Unfortunately, due to time restrictions, the evaluation was realized with only one class of users.The evaluation was carried out with 4 users, one of them individually, and other three users as a group. 

\subsection{Functional test}


\subsection{Non functional test}

\subsection{Design critique}



The development of the system with the final requirements has been carried out. 
The testing method was as follows: An invitation was sent to the UK asthma advisory group inviting participants with at least a basic computational background (As the evaluation is targeting a mobile application, the feedback from non computer users would have been biased), and interest about tracking air quality. 
A video camera was recording the meeting room to gather the opinions and discussion about the system. 
Device logging. 
User logging. 