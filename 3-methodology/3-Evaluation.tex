\section{Evaluation}
Considering that a deliverable of my project dissertation is a mobile application that includes technical and design aspects, the evaluation should be qualitative and quantitative. Quantitative in order to get measurable indicators of how well the user-requirements were met from a software engineering perspective, and qualitative to get an insight from an interaction-design perspective to collect thoughts and reflections about the product.

\subsection{Evaluation with users}
An evaluation with potentially users will be carried out. The participants will required to meet the following criteria:
\begin{itemize}
	\item Basic computer knowledge and confidence using mobile applications.
	\item Interested in air quality data.
\end{itemize}

Even though the application will be designed to be understood by everyone, is important for the user testers to be able to handle computers and smart-phones in a confident way, otherwise they will not be able to complete the tasks within the system independently and may add bias to the gathered results. Apart from this, the attendants should show an interest on air quality to be more likely to be engaged by the applications; and in order to give a more substantial feedback based on previous usage of air quality presentation instruments (computerized or non-computerized).

\subsection{Usability Testing}
It is of particular interest gathering user levels of satisfactions, and measurable indicators of how well the non-functional and functional requirements were met by the application, in other words to asses how well people can understand and use a product or prototype.

"This method is accomplished by identifying representative users, representative tasks, and developing a procedure for capturing the problems that users have in trying to apply a particular software product in accomplishing these tasks" \cite{Scholtz2003} The advantage, as discussed before; is the involvement of the user; however, the test attendants should be a representative group of the potential final users of the product; and the selected tasks should aim to be as realistic as possible because "results are based on actually seeing what aspects of the user interface cause problems for representative users". \cite{Scholtz2003}

\subsection{Design critique}
The aim of including this method, is to gather impressions and thoughts about the system. 

A deeper understanding of a design in its situated context \cite{Frauenberger2013}
"Process of discourse on many levels of the nature and effects of an ultimate particular design". "Comment on the qualities of an ultimate particular from an holistic perspective, including reason, ethics, and aesthetics as well as minute details of form and external effects on culture".\cite{Blevis2007}

Requirements for a critiquing tool


