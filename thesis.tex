%%This is a very basic article template.
%%There is just one section and two subsections.
% \documentclass{article}
%\documentclass[phd,ilcc,twoside]{infthesis}
\documentclass[bsc,logo, abbrevs]{infthesis}
\project{\textbf{Supervisor}: Dr Ewan Klein}

% This package is for times font
\usepackage{times}

% Adjust tables
\usepackage{adjustbox}

% This package is for Informatics thesis style
\usepackage{eushield}

% This package is for your references
\usepackage[numbers]{natbib}

% This package is for figures
\usepackage{graphicx}

% This package is for urls
\usepackage{url}

%Prevents placing floats before a section.
\usepackage[section]{placeins}

% Degree command
\usepackage{gensymb}

% For symbols and equations 
\usepackage{latexsym}
\usepackage{amsmath}
\usepackage{acronym}

% Images below text

\usepackage{float}

% Using urls

\usepackage{url}

%Wrap text around figures

\usepackage{wrapfig}

%Euro symbol

\usepackage{eurosym}

%verbatim color

\usepackage{fancyvrb,xcolor}

%verbatim break lines

\usepackage{spverbatim}

% Referencing in center
\usepackage{csquotes}
\renewcommand{\mkbegdispquote}[2]{\itshape}

% Tables with fixed width

\usepackage{array}
\newcolumntype{L}[1]{>{\raggedright\let\newline\\\arraybackslash\hspace{0pt}}m{#1}}
\newcolumntype{C}[1]{>{\centering\let\newline\\\arraybackslash\hspace{0pt}}m{#1}}
\newcolumntype{R}[1]{>{\raggedleft\let\newline\\\arraybackslash\hspace{0pt}}m{#1}}


% Micrometer symbol

\usepackage{siunitx}

% Quotes 

\newcommand{\quotes}[1]{``#1''}


%% Information about the title, etc.
\title{AirQueue: \protect\\ A Personal Mobile Application to Visualise Air Quality in Scotland}
\author{Alberto Vazquez Martinez}

%% Specify the abstract here.
\abstract{

As air quality over the world becomes crucial for the environmental agenda, new air quality monitoring sensors are being developed and deployed. The gathered data is vaster, more dynamic and coarse-grained than ever before. However, current approaches to disseminating it often fail on being useful, intuitive and personal enough to serve as a decision support tool for people concerned with the quality of the air they breathe. This thesis explores the current approaches used for air quality visualisation as well as the available data sources and technologies to identify how can be improved to be more useful for people. The aim of this dissertation project is to analyse, develop and implement a mobile application to serve such a purpose by taking into account the final users as designers and stakeholders.

}

\begin{document}

%% First, the preliminary pages
\begin{preliminary}

\maketitle
\begin{acknowledgements}

I express my sincere gratitude to my supervisor Dr Ewan Klein for his excellent advice and guidance throughout the project. Without his constant motivation the project would never end up as it did.
Many thanks as well to Catherine Magil for all the time and advice offered. Also, my sincere gratitude to everyone from the Asthma UK Group for the time offered to enlighten the project. Similarly I would like to thank my sponsor institution CONACYT for the finantial 
Finally, I would like to thank the continuous support of my family, that despite the distance have always been by my side.

\end{acknowledgements}

%% Create the table of contents
\standarddeclaration
%%\dedication{Acknowledgements}

\tableofcontents
\listoffigures
\iffalse
\listoftables
\begin{accron}\begin{acronym}[MPC] % Give the longest label here so that the list is nicely aligned
\acro{MPC}{model predictive control}
\acro{TLA}{Three Letter Acronym}
\acro{COMEAP}{Committee on the Medical Effects of Air Pollutants}
\acro{DEFRA}{Department for Environment Food and Rural Affairs}
\acro{NO2}{Nitrogen Dioxide}
\acro{O}[O\textsubscript{3}]{Ozone}
\acro{SO}[SO\textsubscript{2}]{Sulphur Dioxide}
\acro{NO}[NO\textsubscript{x}]{Nitrogen Oxides}
\acro{PM}{Particle Matter}
\acro{PM}[PM\textsubscript{10}]{Particle Matter 10 micrometer}
\acro{PM}[PM\textsubscript{2.5}]{Particle Matter 2.5 micrometer}
\acro{IaaS}{Infrastructure as a service}
\acro{UI}{User Interface}
\end{acronym}\end{accron}
\fi
\end{preliminary}

\chapter{Introduction}

\section{Air Quality in Scotland}
\section{Aims and Objectives}
The first aim of this project is to revise current approaches for air quality data dissemination and visualisation. Furthermore, this project examines the current needs of pollution-sensitive and non-sensitive users towards a new air-quality data dissemination tool in order to accomplish the main goal: Develop and implement a mobile application to visualise air quality data in a useful, and more personal way. 

Another important objective is to take an user-centered approach while accomplishing the main goal, this by bearing in mind the users as co-designers and stakeholders throughout three different design iterations of the product in order to create a final prototype that reflects more accurately the user needs and thoughts.

\iffalse
DATA -> APP -> USEFUL -> PERSONAL -> DECISION SUPPORT -> DESIGNED BY PEOPLE
\fi
\input{1-introduction/3-Challenges}
\section{Dissertation Outline}
This dissertation is structured as follows: 
 
\bigskip
 Chapter 2: Background
\bigskip

This chapter discusses relevant background topics key to understanding the dissertation. It explains the project's importance in the context of the pollution problem, including definitions to follow the basic air quality dissemination terms and related health issues. Past approaches for visualising are also discussed and critically evaluated, as well as explaining in detail what is meant by data visualisation and how to achieve it from a decision support perspective.
 
\bigskip
 Chapter 3: Methodology
\bigskip
  
This chapter describes the methodology and processes that have been used throughout the project. It describes how the development was addressed not only from a human-computer interaction perspective but as well from a software engineering perspective. 
  
\bigskip  
 Chapter 4: Analysis and Design
\bigskip

The analysis and design of the system are  both addressed in this chapter. A detailed analysis of the current needs of a new air quality application is presented, as well as explaining the design and infrastructure choices made for achieving the project goal in a way all stakeholders are considered. The first application prototype is presented.
 
\bigskip
Chapter 5: Implementation
\bigskip

This chapter describes how the required infrastructure and software components were set-up to proceed with the implementation of the final two prototypes, as well as the transitional choices that resulted from the user-centred development approach.

\bigskip
 Chapter 6: Evaluation
\bigskip

The final technical and user evaluations are presented in this chapter by describing how well the defined requirements in Chapter 4 were met with quantitative and qualitative methods.

\bigskip
 Chapter 7: Conclusion
\bigskip

This chapter looks at the aims of the dissertation from an integrative perspective highlighting what has been accomplished so far and giving recommendations for future research.
\chapter{Background}
This section sets the ground for the main topics involved in the project. It establishes a common understanding of the terms used to disseminate air quality; it attempts to study and review critically past approaches of air quality visualisations and defines visualisation from a decision support perspective.
\section{Air Quality}
\subsection{Air pollution and its health effects}
Air pollution can be defined as a group of chemicals present in the atmosphere that are harmful to humans, animals or vegetation. It is mainly caused by human activities, such as transport, industry, or agriculture. However, it can also be influenced by other natural sources. Understanding air pollution is important because many health consequences are the result of high pollution levels. 
\begin{figure}[h]
  \centering
  \includegraphics[scale=.8]{images/great_smog.jpg}
  \caption[Great smog of 1952]{Great smog of 1952 \cite{ElliotWagland2013}}
  \label{fig:interaction_design}
\end{figure}

The adverse health effects of pollution were noted in London's Great Smog of 1952. Thousands of people died in Greater London due to exposure over several days to a highly contaminated atmosphere. Many others became ill or experienced retarded symptoms \cite{Bell2008}. The fog originated from coal burning, vehicle exhaust and other atmospheric factors. Although many human activities which contribute to pollution have changed since then, both the immediate and long-term impacts of pollution continue to be of great concern to human health.

Pollution particles can be categorised into gaseous pollutants, persistent organic pollutants, heavy metals and particulate matter. They vary in their chemical composition, emission sources and impact on health. 

Gaseous pollutants are sulphur dioxide (\SOTWO), nitrogen oxides (\NOX), carbon monoxide (CO), ozone (\OTHREE) and volatile organic compounds (VOCs). The principal source of gaseous pollutants is the combustion of fossil fuels and diesel emission from vehicles. Nitrogen oxides (\NOX) is a general term that includes nitric oxide (NO) and nitrogen oxide (\NOTWO). Gaseous pollutants can affect our health by inflaming the airways and lungs, and in the long term, affect the function of the lungs \cite{AirQualityExpertGroup2004} \cite{WHO2003}.

Particulate matter (PM) is a mixture of solid and liquid particles (such as sulphate, nitrates, ammonia, sodium chloride, black carbon, mineral dust and water). PM are categorised according to their diameter size measured in microns (\SI{}{\micro\metre}, one millionth of a metre). Particles smaller than 10 microns (\PMTEN) are known as coarse particles. Smaller particles with a size of up to 2.5 and 1 microns (\PMTWO and \PMONE) are known as fine and ultra-fine particles respectively. They are differentiated in sizes due to their aerodynamic properties, how they are transported into the air, as well as how far they can get into the respiratory system. According to the World Health Organisation, PM is the most harmful pollutant because it can pass through the nose and throat and enter the lungs. There is also evidence that it is associated with risk of cardiovascular disease \cite{Polichetti2009}. 

Air quality is also affected when pollution particles interact with further complex chemical structures, temperature and humidity. \NOTWO, PM and O\textsubscript{3}  pollutants are transformed by atmospheric processes and thus making it hard to evaluate their individual impact. As an example, ground level ozone is produced when sunlight interacts with \NOTWO and volatile organic compounds. Furthermore, \NOTWO and other nitrogen oxides also contribute to PM generation, making \NOX a particularly concerning pollutant.

\subsection{Air quality data dissemination}
The publication of air quality data aims to better inform and raise awareness amongst citizens who can then make more sustainable and environmental choices but also to look better after their health. According to Thinh and Vogel’s report for the Environmental Informatics and Systems Research journal, \quotes{what was lacking (and it still is), is a model for effective communicating of environmental information to the public} \cite{Thinh2007}. Terms like assessment, limit values, target values and concentration, among others, are commonly used by air quality data publishers to describe the current or forecasted quality status. However, there is not general agreement on how air quality information should be disseminated to the general public in a way that is understood immediately and intuitively by the wider public. 

In general, it is complex to categorise and establish measures for the different components of air pollution due to their heterogeneous nature and the chemical reactions that occur between them. Measurement methods and units vary from institution to institution and regulation standards vary between countries, which may give rise to ambiguity. Furthermore, much of the available data is represented in a tabular format, including various information for individual pollutants, as exemplified in table \ref{tab:pollution_tabular_data}. 

This table was extracted from the  Department for Environment Food and Rural Affairs (DEFRA) website \cite{DepartmentforEnvironmenta}, and shows measures related to the air quality from a sensing station located in Deaconess Garden in the south of Edinburgh. At first glance the table poses key questions for the novice on air-quality trying to crack the data. Firstly, the pollution codes such as PM2.5, PM10, NO2, NOX as NO2 and their subtle differences must be understood. Secondly, some measurement units are tagged with the monitoring method used to extract the information, like the TEOM FDMS \footnote{Which indicates that the sensing methods were Tapered Element Oscillating Microbalance and Filter Dynamics Measurement System \cite{Quality2005}} tag. Lastly, it is difficult to know which measurements are of more interest given varying individual circumstances. As stated by Brimblecombe and Schuepbach \cite{P.Brimblecombe2008}, \quotes{many people complain that the information is unintelligible, while some have even seen it as an attempt of government to blind the public with science}.  All in all, making sense of the terms and values that are used to represent air quality data to the general public tests the layman's understanding.
\begin{table}[ht]
\centering
\begin{adjustbox}{width=1.2\textwidth,center=\textwidth}
\begin{tabular}{rlrrrrrrr}
  \hline
 Pollutant & Date & Time & Measurement & Unit & Period & Comment  \\ \hline
    Ozone (O3) & 20/07/2016 & 07:00 & 63.06412 & µg/m3 & Hourly & - \\
    Nitric oxide (NO) & 20/07/2016 & 07:00 & 2.61933 & µg/m3 & Hourly & - \\
    Nitrogen dioxide (NO2) & 20/07/2016 & 07:00 & 27.34875 & µg/m3 & Hourly & - \\
    Nitrogen oxides as nitrogen dioxide (NOXasNO2) & 20/07/2016 & 07:00 & 31.36500 & µg/m3 & Hourly & - \\
	Sulphur dioxide (SO2) & 20/07/2016 & 07:00 & 14.63495 & µg/m3 & Hourly & - \\
	Carbon monoxide (CO) & 20/07/2016 & 07:00 & 0.081494 & mg/m3 & Hourly & - \\
	PM10 particulate matter (Hourly measured) (PM10) & 18/07/2016 & 15:00 & 10.900 & µg/m3 (TEOM FDMS) & Hourly & - No current data. \\
	Non-volatile PM10 (Hourly measured) (Non-volatile PM10) & 19/07/2016 & 07:00 & 26.700 & µg/m3 (TEOM FDMS) & Hourly & - No current data. \\
	Volatile PM10 (Hourly measured) (Volatile PM10) & 19/07/2016 & 07:00 & 5.500 & µg/m3 (TEOM FDMS) & Hourly & - No current data. \\
	PM2.5 particulate matter (Hourly measured) (PM2.5) & 18/07/2016 & 15:00 & 4.300 & µg/m3 (TEOM FDMS) & Hourly & - No current data. \\
	Non-volatile PM2.5 (Hourly measured) (Non-volatile PM2.5) & 19/07/2016 & 07:00 & 16.300 & µg/m3 (TEOM FDMS) & Hourly & - No current data. \\
	Volatile PM2.5 (Hourly measured) (Volatile PM2.5) & 19/07/2016 & 07:00 & 5.000 & µg/m3 (TEOM FDMS) & Hourly & - No current data. \\
	Modelled Wind Direction (Dir) & 19/07/2016 & 24:00 & 50.6 & \degree & Hourly & - No current data. \\
	Modelled Wind Speed (Speed) & 19/07/2016 & 24:00 & 6.2 & m/s & Hourly & - No current data. \\
	Modelled Temperature (Temp) & 19/07/2016 & 24:00 & 14.6 & °C & Hourly & - No current data. \\
	PM10 Ambient Temperature (AT10) & 19/07/2016 & 07:00 & 19.4 & °C & Hourly & - No current data. \\
	PM10 Ambient pressure measured (AP10) & 19/07/2016 & 07:00 & 989.0 & mb & Hourly & - No current data. \\
	PM2.5 Ambient Temperature (AT25 ) & 19/07/2016 & 07:00 & 17.5 & °C & Hourly & - No current data. \\
	PM2.5 Ambient Preasure (AP25) & 19/07/2016 & 07:00 & 988.0 & mb & Hourly & - No current data. \\
   \hline
\end{tabular}
\end{adjustbox}
\caption{Air quality tabular data representation. \cite{DepartmentforEnvironment}}
\label{tab:pollution_tabular_data}
\end{table} 

\subsection{The COMEAP air quality index and health advice}
In order to understand the correlation between air quality data and its effects on human health, the Committee on the Medical Effects of Air Pollutants (COMEAP) developed the air quality index based on health evidence. It is used to communicate real-time air quality levels and their short-term health effects for five selected harmful pollutants: particulate matter (\PMTEN and \PMTWO), ozone (\OTHREE), sulphur dioxide (\SOTWO), and nitrogen dioxide (\NOTWO) as shown in Figure \ref{fig:air_quality_index}. The index employs a colour scale and a number  to inform the concentrations of each specific pollutant. Four bands are employed: Low, Moderate, High and Very High. 

\begin{figure}[H]
\begin{adjustbox}{width=1\textwidth,center=\textwidth}
  \centering
  \includegraphics[scale=.8]{images/air_quality_index.png}
\end{adjustbox}
  \caption[The COMEAP air quality index]{The COMEAP air quality index \cite{HealthProtectionAgencyfortheCommitteeontheMedicalEffectsofAirPollutants2011}}
  \label{fig:air_quality_index}
\end{figure}

There is substantial evidence that the elderly, children, and people who suffer from chronic diseases such as asthma are in greater danger of suffering symptoms and health consequences from lower pollution concentrations than the general public \cite{Koenig1999} \cite{Kampa2008} \cite{Zones2010}. The COMEAP includes such population in their Air Quality Index to give such individuals the opportunity to modify their behaviour and reduce the severity of their symptoms. Furthermore, the air quality index is accompanied by health advice (Figure \ref{fig:air_quality_health_advice}) that provides specific health messages targeting both population groups, sensitive and non-sensitive providing information about the actions that should be taken to avoid symptoms and health effects.

\begin{figure}[H]
\begin{adjustbox}{width=1\textwidth,center=\textwidth}
  \centering
  \includegraphics[scale=.8]{images/air_quality_health_advice.png}
\end{adjustbox}
  \caption[Air quality health advice]{Air quality health advice \cite{HealthProtectionAgencyfortheCommitteeontheMedicalEffectsofAirPollutants2011}}
  \label{fig:air_quality_health_advice}
\end{figure}


\subsection{Air quality sensors}

\begin{itemize}

\item Fixed sensors: 

Fixed sensors are installed and maintained by the UK government and local authorities in Scotland. These are of two kinds, automatic sensors from the Automatic Urban and Rural Network (AURN)\footnote{\url{https://uk-air.defra.gov.uk/networks/network-info?view=aurn}}, and non-automatic sensors maintained by Scottish local councils.

	\begin{itemize}
    
    \item Automatic sensors\footnote{\url{https://uk-air.defra.gov.uk/networks/network-info?view=non-automatic}} produce hourly concentrations and data is sent automatically over the network. Their purpose is to check that EU and other regulatory standards are being met as well as informing the public about air quality. These sensors monitor a wide range of pollutants (\NOX, \SOTWO, CO\textsubscript{2}, O\textsubscript{3}, \PMTWO and \PMTEN, as well as temperature and humidity) which are later collected and processed by Ricardo Energy and Environment\footnote{\url{http://ee.ricardo.com}} and presented on HTML webpages at the Scottish Air Quality website\footnote{\url{http://www.scottishairquality.co.uk/}}. Alternatively, CSV files are provided on demand. The advantage of these sensors is their accuracy as well as the automatic workflow of the data. On the other hand, having few sensors spread around the city does not offer much resolution to current limitations. 
    \item Non-automatic sensors\footnote{\url{https://uk-air.defra.gov.uk/networks/network-info?view=non-automatic}} are commonly known as diffusion tubes. They offer higher resolution as there are over 100 situated in Edinburgh. The drawback is that samples must be collected physically, lessening the time at which the information can be communicated. Also, the data gathered is only limited to \NOTWO readings.
    \end{itemize}

\item Portable sensors: Their various advantages are that they can be carried throughout the day and allow for more personal and accurate readings based on the actual location of the user. The drawback is that some of them are limited to one pollutant, and the ones that can measure a range of different pollutants are still expensive. Examples are the Air Beam\footnote{\url{http://aircasting.org/}}, the Clean Space Tag\footnote{\url{https://store.clean.space/}} and the Libelium Gases PRO\footnote{\url{http://www.libelium.com/development/waspmote/documentation/gases-pro-board-technical-guide/}}. The first is open source and able to measure particle matter. The second is a proprietary project which only includes carbon monoxide readings and costs around 100\pounds. The latter measures a wider range of pollutants (CO, CO\textsubscript{2}, O\textsubscript{3}, \NOX and other gases) at a higher price (approximately \euro{}2000). 

\item Participatory sensors:

Some projects such as CitiSense \cite{Nikzad2012} include users as 'human sensors' by reading their perceptions towards air quality in certain locations. This method aims to get more fine-grained information about air quality and engage the citizens in the pollution problem. 

\end{itemize}




\section{Data visualization}

\subsection{Definition}
A number of definitions have been proposed to explain what data visualisation means. According to Friendly \cite{Friendly2009} the term arose together with the birth of statistical thinking, being a way to illustrate mathematical language, their trends, tendencies and distributions trough the use of diagrams and graphics. Few \cite{StephenFew2013} describes it as a graphical display of abstract information for the purpose of sense-making and communication. 
However, the definition does not exclude representations that are not algorithmically drawn. The definition offered by Iliinsky and Steele \cite{Iliinsky2011} is more specific: 
\begin{displayquote}
Any visual representation of data that is:
	\begin{itemize}
	\item algorithmically drawn (may have custom touches but is largely rendered with the help of computerized methods);
	\item easy to regenerate with different data (the same form may be repurposed to represent different datasets with similar dimensions or characteristics);
	\item often aesthetically barren (data is not decorated); and
	\item relatively data-rich (large volumes of data are welcome and viable, in contrast to infographics).
    \end{itemize}
\end{displayquote}
\iffalse
\begin{figure}[h]
  \centering
  \begin{adjustbox}{width=.8\textwidth,center=\textwidth}
  \includegraphics[scale=1]{images/air_pollution_infographic.jpg}
  \end{adjustbox}
  \caption[Air pollution infographic]{Air pollution infographic \cite{NiallMcCarthy}}
  \label{fig:air_pollution_infographic}
\end{figure}
\fi
A visualization helps towards understanding data  by taking advantage of the human visual system to process a large amount of information quickly, thus allowing the human brain to identify patterns, links and relationships between the represented objects. Daniel et al. \cite{KeimDaniel2010}, also states that visualizing enables people to: 
\begin{displayquote}
	\begin{itemize}
\item  Synthesise information and derive insight from massive, dynamic, ambiguous, and often conflicting data.
\item Detect the expected and discover the unexpected.
\item Provide timely, defensible, and understandable assessments.
\item Communicate these assessment effectively for action.
	\end{itemize}
\end{displayquote}

Data visualization is essential when the available data is vast and dynamic, and when raw data does not make sense by its own. Therefore, data must be encoded using technological, and design elements;  potentially making use of disciplines such as statics, data mining, human computer interaction and graphic design. 

\subsection{Designing data visualizations}
The Figure \ref{fig:data_visualization_process} depicts the visual analysis process according Daniel et al. \cite{KeimDaniel2010}. In the first stage,  data should be collected, (usually from many distinct data sources) and standardised into one common format. This later enables us to choose between the creation of models, or visualisations; models are an automated representation that requires data mining techniques; whereas visualisations are manual representations that can be created through simple mapping from data to the visual context. Both representations are linked together, to enable validation and refinement through iteration. The final stage; which is the one we are more concerned with is the knowledge gained from the representation; which will aim to respond the questions for which they were created in the first place.
\begin{figure}[!htb]
\begin{adjustbox}{width=1\textwidth,center=\textwidth}
  \centering
  \includegraphics[scale=1]{images/data_visualization_process.png}
\end{adjustbox}
  \caption[The data visualization process]{The data visualization process \cite{KeimDaniel2010} }
  \label{fig:data_visualization_process}
\end{figure}

According to Ben Fry \cite{Cleveland1993}, one specific approach for data visualisations is illustrated in Figure \ref{fig:data_visualization_stages}, which establishes a common series of steps that can be followed for this purpose:

\begin{displayquote}
	\begin{itemize}
\item Acquire: Obtain the data, whether from a file on a disk or a source over a network. 
\item Parse: Provide some structure for the data’s meaning, and order it into categories. 
\item Filter: Remove all but the data of interest. 
\item Mine: Apply methods from statistics or data mining as a way to discern patterns or place the data in mathematical context.
\item Represent: Choose a basic visual model, such as a bar graph, list, or tree. 
\item Refine: Improve the basic representation to make it clearer and more visually engaging. 
\item Interact: Add methods for manipulating the data or controlling what features are visible.
\end{itemize}
\end{displayquote}

\begin{figure}[!htb]
\begin{adjustbox}{width=1\textwidth,center=\textwidth}
  \centering
  \includegraphics[scale=1]{images/visualization_stages.png}
\end{adjustbox}
  \caption[Visualization stages]{Visualization stages  \cite{Cleveland1993} }
  \label{fig:data_visualization_stages}
\end{figure}
From this perspective, it is possible to establish that all visualisations require an elemental process guided by the designer to enable the recognition of patterns within the data. Also that the design process is an iterative task where the designer may go back to redefine any stage of the process until achieving the desired result. 

\subsection{Data visualizations for decision support}
\quotes{A decision-making process usually includes the acquisition of related information, the construction of a mental representation of the problem and solutions, and the identification of an optimal solution} \cite{carroll1987mental} as cited in \cite{Zhu2008}, decision making involves acquiring domain-specific information about a particular question which is the intended output of a data visualisation. It is, therefore, important to understand the effects of data visualisations in decision support making, if any.

The decision-making process takes into account the decision maker's skill, the decision task, and the problem space. In our specific context, the decision maker, should be able to understand the data in its represented space (visualisation), evaluate the problem (avoid air pollution), and execute a decision task (leave the polluted space). \quotes{A well-designed visualisation takes features of the decision task and the characteristics of the decision makers into consideration} \cite{Zhu2008}. Thus, visualisations should be created in the specific context of the decision makers; discovering beforehand the knowledge they should have; the problems they are likely to face; and the potential decisions they should be able to take. 

\section{Air quality visualizations}
The problem of making the air quality problem visible to the general population has been addressed before, from map-based representations to wearable devices and in-city displays. These approaches contemplate new or known ways to tackle the problem of air quality, and therefore; making the invisible visible.

\subsection{Table based}
The most basic representation of air quality data is as tabular data, generally including some colours to indicate the quality level of the measurement as shown in figure \ref{fig:table_based_visualization}. From these kinds of representations is easy to read information from multiple places. One drawback is that much information is available for the reader to process easily, not always pertinent to the reader current location or requirements. Another issue is that the user requires a previous understanding of the terms and measurements used to annotate the table like \quotes{hourly mean} or \quotes{runing 8 hour mean}. 

\begin{figure}[H]
\begin{adjustbox}{width=1\textwidth,center=\textwidth}
  \centering
  \includegraphics[scale=1]{images/tabular_data.png}
\end{adjustbox}
  \caption[Tabular visualization]{Tabular visualization \cite{DepartmentforEnvironment}}
  \label{fig:table_based_visualization}
\end{figure}

\subsection{Map based}
Through maps people can understand where measurements are being made, they are also good to discover pollution-safe routes when walking around the city or to find out which spots in the city are well suited to realise physical activity.  Some representations include colour indicators to show how good or bad the measurements and can be enabled to discover history data by selecting previous dates. However; it is unlikely that one is interested in the measurements for all Scotland's regions, and it is hard to visualise correctly overlaid numeric data on small screens.

\begin{figure}[H]
\begin{adjustbox}{width=1\textwidth,center=\textwidth}
  \centering
  \includegraphics[scale=.30]{images/map_visualization.png}
\end{adjustbox}
  \caption[Map-based visualization]{Map-based visualization \cite{Scottishairquality.co.uk2016}}
  \label{fig:web_based_desktop_visualization}
\end{figure}


\subsection{Line based}
Line graphs serve well to represent variations of data in defined periods of time allowing to depict patterns and predictions within the data quickly. It is also possible to include various datasets in one single graph, enabling comparison between them. The inAir project \cite{Kim2013} utilises a line graph to represent indoor particle matter count over time. Their findings suggested that this approach allows users to reflect on their behaviours and air quality status, persuading them to modify their practices to decrease their exposure to air pollution. 

\begin{figure}[H]
\begin{adjustbox}{width=1\textwidth,center=\textwidth}
  \centering
  \includegraphics[scale=1]{images/InAir.png}
\end{adjustbox}
  \caption[inAir project: line-based visualizations]{inAir: line-based visualization\cite{Kim2013}.}
  \label{fig:line_based_inAir}
\end{figure}


\subsection{Photo based}
Photo-based representations as proposed by Lin \cite{Lin2014} allow people to understand levels of pollution by taking pictures of the current environment. The pictures are later adjusted using a filter that represents the air quality status (NO2) at that location. This serves as a playful way to visualise air quality while performing a habitual activity as it is taking pictures. On the other hand, this project only aggregates nitrogen dioxide readings, excluding other pollutants that may be pertinent, as well as not going further to indicate what should a person do given the pollution levels.

\begin{figure}[H]
\begin{adjustbox}{width=.8\textwidth,center=\textwidth}
  \centering
  \includegraphics[scale=.4]{images/instaNO2.jpg}
\end{adjustbox}
  \caption[InstaNO2 project: photo-based visualizations]{InstaNO2: photo-based visualizations \cite{Lin2014}.}
  \label{fig:photo_based_instaNO2}
\end{figure}

\subsection{Tangible approaches}
Other approaches have made use of tangible objects to visualise air pollution in a more engaging way. The Human sensor project \cite{InvisibleDust2016} employs wearable art and in-city performances to reveal changes in pollution. Similarly, WearAir, \cite{Kim2010} is an expressive T-shirt that senses VOCs and expresses the readings making use of embedded lights. Kuznetsov et al. \cite{Kuznetsov2011} developed glowing balloons with air quality sensors for users to explore urban air quality. The \textit{IBM think} exhibit \cite{IBM2012} is a 123-foot long digital wall display located in New York city which renders art patterns and real time streaming visualisations showing air quality data as well as traffic flow, energy, and water usage. This is intended to be an immersive in-city experience for users to explore and understand the role of data in the world. Tangible approaches are useful because the evoke curiosity and surprise while gaining an insight of what is happening. In contrast, they are not intended for personal day to day usage but as a way of expression and art. 

\begin{figure}[H]
\begin{adjustbox}{width=.8\textwidth,center=\textwidth}
  \centering
  \includegraphics[scale=.4]{images/think_human_sensor_balloons.jpg}
\end{adjustbox}
  \caption[Tangible visualizations]{The human-sensor on the top-left \cite{InvisibleDust2016}, glowing balloons on the top-right \cite{Kuznetsov2011}, and IBM think on the bottom \cite{IBM2012}.}
  \label{fig:photo_based_instaNO2}
\end{figure}


\section{Mobile applications}
In recent years mobile applications are becoming increasingly popular in many domains, such as business, health and entertainment. According to the 2015 mobile app report \cite{ComScore}, the time spent on mobile devices grew up from 51 percent in share spent time to a 62 percent, leaving the share spent time of desktop on a 38 percent and becoming the number one of digital media consumption in 2015. Also, the usage of apps which include health or tracking functions was unknown until 2014, when several apps experienced a huge grow of up to 922 percent each year. 

\subsection{Personal applications and new hardware capabilities}

Mobile devices have reached a point where their hardware capabilities are comparable to a desktop computer capabilities regarding their processing power and available RAM; furthermore, they provide added functions that are unimaginable for a desktop computer, such as GPS, accelerometer; and the ability to be carried throughout the day. This enables new ways of living and interacting with technology as well as bringing closer the idea of ubiquitous computing, "perhaps ubiquitous computing is already here, but took a form other than that which had been envisioned." \cite{Bell2007}. 

\begin{figure}[h]
\begin{adjustbox}{width=.4\textwidth,center=\textwidth}
  \centering
  \includegraphics[scale=.5]{images/sleep_tracking.jpg}
\end{adjustbox}
  \caption[Fitbit application sleep tracking]{Fitbit application sleep tracking \footnote{\url{https://www.google.com/fit/}}}
  \label{fig:google_fit}
\end{figure}

Example of emerging applications that heavily rely on new devices hardware's capabilities, are Google Fit \footnote{\url{https://www.google.com/fit/}}, Nike + \footnote{\url{http://www.nike.com/us/en_us/c/nike-plus/running-app-gps}}, Fitbit \footnote{\url{https://www.fitbit.com/uk}} Jawbone Up \footnote{\url{https://jawbone.com/up}} and Garmin Connect \footnote{\url{https://connect.garmin.com/en-US/}}. These apps make use of sensors that although simple and cheap; are able to measure a range of activities carried out by a person. Such activities go from time and quality of sleep, calories ingested during the day, average heart rate of a run or steps taken during a city walk that according Morrison et al. \cite{Rooksby2014},  allows to count and measure areas of a person's life to optimize behavior as desired.

Recent research \cite{Barkhuus2011} suggests that people uses mobile applications in personalized or individual manners, adapting functions to meet their priorities; adding  new functions to create their own unique experiences based on their everyday lives. There is an enormous opportunity for crafting new applications that not only can give unique and valuable experiences; but influence the way people do things in the real world. 

\subsection{Challenges and usability considerations}

Core differences between desktop and mobile applications make mobile development more challenging than desktop development \cite{Chittaro2006}. Important examples of this that should be accounted when designing visualizations for mobile devices are, among others; limited processing power, smaller screen, multiple types of screens and slower connectivity. This issues become evident when trying to translate a visualization such as the shown in 
\iffalse
\ref{fig:web_based_desktop_visualization}, 
\begin{figure}[h]
\begin{adjustbox}{width=1.2\textwidth,center=\textwidth}
  \centering
  \includegraphics[scale=.5]{images/visualization_desktop_example.png}
\end{adjustbox}
  \caption[Line-based visualization]{Line-based visualization}
  \label{fig:web_based_desktop_visualization}
\end{figure}
\fi
It is particularly challenging to design mobile applications in contrast to desktop applications because 

(PEND)





\iffalse
\section{Air quality sites and applications}
Currently there are several air-quality related web or mobile applications. In order to understand how to create a new application
with added value and functionalities, a review on existing applications or was carried out. Due to limited time, this search was restricted to software working in Scotland on Android devices or web browsers. More specifically, the more popular websites and android applications in the market and in search engines were analyzed.
The following table illustrates the functions that are currently offered by this applications or websites. 

\begin{itemize}
	\item F1 (Air quality info in general)
	\item F2 (Air quality in many locations)
    \item F3 (Air quality in current location)
    \item F4 (Air quality in form of a map)
    \item F5 (Pollution indicators)
	\item F6 (Notifications)
    \item F7 (Changes over time)
	\item F8 (Forecast)
    \item F9 (Personal sensor)
\end{itemize}

\begin{table}[ht]
\centering
\begin{adjustbox}{width=1.2\textwidth,center=\textwidth}
\begin{tabular}{rlrrrrrrrrrr}
  \hline
 Name & Type & Description & F1 & F2 & F3 & F4 & F5 & F6 & F7 & F8 & F9 \\ \hline
    \url{http://www.scottishairquality.co.uk/} & Website & Air quality status and forecasts offered by the Scottish Government & - & X & X & X & X & X & X \\
    \url{http://www.environment.scotland.gov.uk/} & Website & Air quality status and forecasts offered by the Scottish Government & - & X & X & X & X & X & X \\
    \url{http://www.environment.scotland.gov.uk/} & Website & Air quality status and forecasts offered by the Scottish Government & - & X & X & X & X & X & X \\
    \url{http://www.environment.scotland.gov.uk/} & Website & Air quality status and forecasts offered by the Scottish Government & - & X & X & X & X & X & X \\    

   \hline
\end{tabular}
\end{adjustbox}
\caption{Air quality tabular data representation. \cite{DepartmentforEnvironment}}
\label{tab:pollution_tabular_data}
\end{table} 
\fi



\section{Issues and current needs}
From outlining why pollution matters, its relevance to different population groups and the use of visualisation to convey relevant data, the limitations of current approaches as a means of communication become apparent. Current approaches fail to take into account the general population and sensitive pollution users according to the COMEAP. There is the need for a tool that includes both user groups and their personal circumstances in order to provide more accurate information for the purpose of decision making. Apps are enabled to create new enhanced personal experiences and may be an appropriate vehicle to achieve such need.

\chapter{Methodology}
In order to accomplish the project aims, it is imperative to select a framework which allows a user-centered development process as well as a flexible and iterative workflow. The project followed an interaction design model and an agile development methodology. The following sections will give a brief outline of these methods, as well as describe how they have been used throughout the project.
\section{Interaction Design Process}
Interaction design can be briefly defined as "designing interactive products to support people in their everyday working lives" \cite{Sharp2011}. To develop a useful product an understanding of what is needed must be established before and during the development process. This brings new challenges and questions, such as if users understand what they are expecting from a specific product, and correctly establishing who the users are beforehand. The interaction design process is a strong user-centred methodology that when correctly carried out will produce an output that reflects the real user's voice and needs. 

Users are the people for whom the system is developed and who will employ it for their goals. Stakeholders are the people that are involved in the development process, influencing the system requirements. For the purposes of this project, the users are both the general public who are concerned with air and its impact on their health status and sensitive users which are at greater risk of suffering health consequences from air pollution. The stakeholders are the secondary agents who have provided an opinion on the capabilities of the system, in this case my project supervisor and other researchers in the field that can give formative feedback.

As described in Figure \ref{fig:interaction_design}, the interaction design process is carried out at four stages: \begin{itemize}
  \item Identify user needs and establishing requirements
  \item Develop alternative designs
  \item Build interactive versions
  \item Evaluate with users
\end{itemize}

\begin{figure}[h]
  \includegraphics[scale=.8]{images/interatcion-design.png}
  \caption[Interaction design process]{Interaction design process \cite{Sharp2011}}
  \label{fig:interaction_design}
\end{figure}

Identifying needs and establishing requirements is the first and most crucial activity of the process. Its objective is to learn and understand the user's needs and communicate them to the developer. According to the Chaos report \cite{Group1994}, problems involving requirements account for more than the 30\% of project failures including lack of user input, incomplete requirements and specifications, or changes in requirements and specifications. Also, the interdisciplinary nature of requirements elicitation and contradictions between stakeholders add complexity to this part of the process. To make this stage more accurate and disciplined it is useful to combine different techniques and methodologies, such as questionnaires, interviews, introspection and brainstorming \cite{Coulin2005}. The selected methods were questionnaires and interviews. Interviews allowed for casual conversations with few potential application users, whereas questionnaires allowed for different points of views from many potential users of the application. 

A support tool for developing alternative designs is the prototype, a physical or digital outline of a screen or task supported by the product. This allows for different purposes: first, as support for the creative process, allowing the designer to print the expected interface and how it will support the intended use cases. Second, as a communication tool for the stakeholders, it supports the flow of ideas between the designer and the people involved in the development. Finally, it allows the designer to get measurable feedback of the capabilities of the product. Prototypes were designed and evaluated in different iterations across the project.
\section{Agile methodology}
Because a piece of software was developed, the development process followed an agile methodology as a good software engineering practice. The benefits of this approach over other ones is that more focus is paid to the product than to the management process, this proves to be a strong advantage when working on constantly-changing and time-constrained projects such as my dissertation project.
The agile manifesto \cite{Martin2002} establishes four values to guide a development process: 
\begin{itemize}
  \item Individuals and interactions over processes and tools
  \item Working software over comprehensive documentation
  \item Customer collaboration over contract negotiation
  \item Responding to change over following a plan. 
\end{itemize}

More grounded, choosing this methodology as a software development approach means that there is no intention to deliver full documentation, manuals or specifications, but that the working software will be delivered with its different phases and prototypes, and that there is a plan up to some extent, including a calendar of deliverables, and management tools are used to guide the development.

\subsection{Project plan}

\subsection{Self-management}

\section{Evaluation}
Considering that a deliverable of my project dissertation is a mobile application that includes technical and design aspects, the evaluation should be qualitative and quantitative. It should be quantitative to get measurable indicators of how well the user-requirements were met from a software engineering perspective, and qualitative to get an insight from an interaction-design perspective to collect thoughts and reflections about the product.

\subsection{Technical evaluation}
A technical evaluation will ensure that the software behaves as expected during deployment. This will be done by inspecting and debugging the code through the entire development process. Code inspections are performed by visually inspecting the program coded statements and allow to discover logic and coding errors to spend less time and effort debugging \cite{Myers2011}. The coding errors that are not discovered on the inspection process will show up when executing the program and may be fixed by debugging. Debugging allows to find and correct suspected errors within the program in a effort-less manner by using debugging tools integrated into modern development editors. Once the software development has been exhausted automated tests will be used to discover errors that still may be present.

\subsection{Evaluation with users}
Even though the application will be designed to be understood by everyone, is important for the user testers to be able to handle computers and smartphones in a confident way. Otherwise, they will not be able to complete the tasks within the system independently and may add bias to the gathered results. 

It is of particular interest to gather user levels of satisfactions, and measurable indicators of how well the non-functional and functional requirements were met by the application, in other words, to assess how well people can understand and use a product or prototype.

\quotes{This method is accomplished by identifying representative users, representative tasks, and developing a procedure for capturing the problems that users have in trying to apply a particular software product in accomplishing these tasks.} \cite{Scholtz2003} The advantage is the involvement of the user; however, the test attendants should be a representative group of the potential final users of the product. The selected tasks should aim to be as realistic as possible because \quotes{results are based on actually seeing what aspects of the user interface cause problems for representative users}. \cite{Scholtz2003}

\subsection{Design critique}
The objective of including this method is to gather impressions and thoughts about the system and leave open space to discover what future releases should aim for. According to Blevis et al. \cite{Blevis2007} it allows to understand the development from the perspective of the user. In this case, will try to understand how the users would include such development in their daily lives. For instance, if they would be willing to use it on a regular basis, or if they would use it because its attractive and fun to use, or just because they feel it would benefit their health in some way. All of these would give a wider understanding of the flaws and strengths of the system.

\iffalse
\quotes{Process of discourse on many levels of the nature and effects of an ultimate particular design}. \quotes{Comment on the qualities of an ultimate particular from an holistic perspective, including reason, ethics, and aesthetics as well as minute details of form and external effects on culture}.\cite{Blevis2007}
\fi


\iffalse\section{Out of Scope Issues}\fi
\chapter{Analysis and Design}
The focus of this chapter is to establish and discuss what the software is going to do and how it is going to achieve its intended goals. The defining actors were from different institutions related to the issues covered by the project. This chapter describes how they were involved as stakeholders, co-designers and evaluators. 

\begin{figure}[h]
\begin{adjustbox}{width=1\textwidth,center=\textwidth}
  \centering
  \includegraphics[scale=1]{images/stakeholders.png}
\end{adjustbox}
  \caption[Colaborators in the project]{Collaborators in the project.}
  \label{fig:stakeholders}
\end{figure}

\section{Users and Stakeholders}

The following users were depicted for this development:

\begin{itemize}
    \item General public: Members of the general public who are concerned with the air quality of their environment.
    \item Pollution-sensitive public: People at risk of suffering health consequences due to air pollution.
\end{itemize}

To gain input from different perspectives, the following stakeholders were involved in the development: 

\begin{itemize}
	\item Researchers 
    \begin{itemize}
      \item Asthma UK researchers: Meetings were arranged with various researchers from the Asthma UK group. They were involved early in the design of a first prototype, by providing ideas and highlighting issues that may arise from the perspective of a a medical researcher. The researchers were:  Dr Aziz Sheikh, director of the research group, as well as Dr Soyiri Ireneous, and Dr Lynn Morrice. They are also members of the Usher Institute of Population Health Sciences and Informatics\footnote{\url{http://www.ed.ac.uk/usher}}
      
    The following issues were raised in the meetings:
      \begin{itemize}
          \item Based on their experience further considerations should be taken into account when designing applications that rely on user input.
          \item Review past approaches/work to avoid known problems.
          \item An application should give actionable feedback rather than just displaying information.
          \item Ask the users in order to find out what they want.
      \end{itemize}
      \item University of Edinburgh School of Informatics researchers: My supervisor Dr Ewan Klein and Catherine Magill, who are both researchers at the University of Edinburgh and part of the Edinburgh living lab which focus is on open data and how it impacts on behaviour and society. They also have a strong background in participatory design and were asked to provide feedback on inner prototypes.
	\end{itemize}
	\item Domain expert  
   \begin{itemize}
      \item Chief Technology Officer: Simon Chapple as CTO of Datalytics technology, a company with expertise in developing mobile applications for a different range of platforms. His opinion was valuable on technical or design issues that may arise in the development. He was asked for feedback about the design of a prototype of the application.
    The following recommendations were raised in the meeting:
    \begin{itemize}
        \item User interfaces should aim to be self-explanatory. 
        \item Colours should be consistent and provide meaning, as users associate colours to real world entities.
        \item Further consideration should be taken when choosing a native vs. hybrid development approach.
        \item Welcome screens should have an immediate, understandable objective.   
    \end{itemize}
\iffalse
\item How to ref this?: Mic starbuck, an activist who is involved in the guverment council and suffers from asthma was involved in the very first ideas of the design of the application, from his input, we were able to understand in a broad sense, what might an asthma sufferer need. (PEND)
\fi
\end{itemize}
    
%"I wonder how you did the animations behave that way"
    \item Users as stakeholders: The final users were involved in three stages. In the requirements elicitation, as co-designers for the prototypes and as evaluators of the final product. 
    \begin{itemize}
            \item Asthma UK patient group: They represent a sample of pollution-sensitive users. They are potentially a good source of data because they are already aware of the issues around air quality. They were involved in two phases: requirements elicitation, which was carried out via an online questionnaire, and the evaluation of the third final prototype.
            \item University of Edinburgh students: They represent a sample of the general population and were involved in three phases: the requirements elicitation, which was carried out via an online questionnaire, the evaluation of inner prototypes and the evaluation of the third final prototype. 
    \end{itemize}
\end{itemize}

\section{Requirements Elicitation}
According to Somerville there are two kinds of requirements \cite{Sommerville2010}:
\begin{displayquote}
\begin{itemize}

\item Functional requirements: statements of services the system should provide, how the system should react to particular inputs, and how the system should behave in particular situations

\item Non functional requirements: Constraints on the services or functions offered by the system .... often apply to the system as a whole, rather than individual system features or services.


\end{itemize}
\end{displayquote}

The main functional requirements were elicited from the users because they were the first reason of the development. This was done through an online Survey Monkey questionnaire,\footnote{\url{https://www.surveymonkey.com/results/SM-Z7HH28LM/}} sent to Asthma UK volunteers and to students from the University of Edinburgh. Feedback from researchers and domain experts was taken into account as non-functional requirements as they did not ask for any particular function per se but just established some guidelines and features they believed would strengthen the application.

\subsection{Survey}

When designing the questionnaire there was an interest in knowing about existing air quality data sources used by both kinds of users and what did they find useful from them. They were also queried on how often they were using the information sources, their ages, and their needs for a new air-quality application.

There were 11 responses in total, 8 from the patient group and 3 of them from the general public. The ages of the respondents were varied: three between 25-34, 3 between 65-74, 2 between 18-24, 2 between 35-44 and 1 between 55-64 years old as shown in the figure \ref{fig:survey_ages}. 

\begin{figure}[H]
\begin{adjustbox}{width=1\textwidth,center=\textwidth}
  \centering
  \includegraphics[scale=1]{images/ages_survey.png}
\end{adjustbox}
  \caption[Survey respondents ages ]{Survey respondents ages.}
  \label{fig:survey_ages}
\end{figure}

Regarding previous usage of air quality sources, only half of the pollution-sensitive respondents were able to name the websites or applications they were currently using, while all the non-sensitive respondents named at least one source of information. Also, the sensitive respondents who reported the use of air quality sources tended to use them on a more regular basis than non-sensitive users. Reflecting on these answers: sensitive users may be inclined to require a support tool on a regular basis, but not all of them are aware of its existence.

The main findings from the survey were the potential new functionalities that users would like in a new Air Quality application, suggesting that there are unaddressed needs from current AQ sources. These are the main focus of the development of a new AQ application. From Figure \ref{fig:survey_new_features} we can see that the most important need is to visualise different individual pollutants, as current approaches often fail on treating them individually and thus enable respective decisions on this information. The second most important unaddressed need is to be able to personalise the air quality advice given a user's personal circumstances. The third most important requirement is to compare and visualise individual pollutants over different periods of time. 

Another interesting outcome was whether including tools to allow users to keep track of their symptoms would be of interest. Just three of the eleven queried persons responded positively suggesting that they may feel it unnecessary or tedious to input their symptoms on a regular basis.


\begin{figure}[H]
\begin{adjustbox}{width=1\textwidth,center=\textwidth}
  \centering
  \includegraphics[scale=1]{images/new_features.png}
\end{adjustbox}
  \caption[New features for an AQ application]{New features for an AQ application.}
  \label{fig:survey_new_features}
\end{figure}

\subsection{Interviews}
 
Meetings were held with the aforementioned researchers and domain experts to receive input from different points of view. At an early stage of the design process, the first person who showed interest was Mic Starbuck. From his perspective as a person suffering from asthma he mentioned the need for a tool to be able to visualise the relative impact upon a person's health based on different factors: like a person's health status, location, altitude and individual pollutants. He also mentioned that he browsed the web frequently to understand how AQ is behaving at different points of the day to decide whether to go outside for certain activities. 

Later in the process, the Asthma UK Centre for Applied Research (AUKAR) showed interest and a meeting was arranged. It took place with Dr Aziz Sheikh, Lynn Morrice, Dr Soyiri Ireneous and Catherine Magill. They expressed interest in collaborating with this project as well as participating closely to the School of Informatics in future research. They were presented with mock-ups of an application with visualisation and tracking purposes. However, based on previous work on tracking asthma status they pointed out that applications that rely on user input may not be successful because they demand too much from the user. For instance, much effort would be required to develop interfaces suited for particular users and their sets of circumstances. From this more focus was put into visualisation and recommendation capabilities than tracking.

Another meeting was held with Dr Soyiri Ireneous, an asthma researcher, and Catherine Magill. The purpose of this session was to understand in more depth the consequences of air pollution upon a person's health and how they might be presented in an air quality application. The possibility of including health advice was discussed. Data  with an easy to follow health guide is more valuable than data alone because the user can immediately know what to do under special circumstances. Also, Dr Soyiri as a mobile phone user, was concerned with the speed of the application as well the memory and battery usage.

\subsection{Functional Requirements}

As time was limited the development covered the first three and most important needs set out by the final users. The following list summarises the final formal functional requirements for the development:

\begin{itemize}
    \item Being able to visualise the air quality status in general.
    \item Being able to visualise individual pollutants status and information on how they impact on a person's health.
    \item Being able to visualise individual pollutants over time.
    \item Being able to receive advice on the air pollution based on the variables like location, age and pollution sensitivity. 
    \item Being able to personalise the received advice indicating the personal pollution tolerance or sensitivity.
\end{itemize}

\subsection{Non Functional Requirements}

Summarising from the meetings, the following were established as non-functional requirements:

\begin{itemize}
    \item Is adequate for mobile devices (\textbf{Mobility}).
    \begin{itemize}
        \item Does not consume much storage space.
        \item Does not consume much battery.
        \item Is compatible with multiple devices.
    \end{itemize}
    \item Is usable (\textbf{Usability}):
    \begin{itemize}
        \item Easy to learn.
        \item Attractive to use.
    \end{itemize}
    \item Is efficient (\textbf{Performance}).
    \begin{itemize}
        \item Starts rapidly
        \item Loads and displays visualisations fast
    \end{itemize}
\end{itemize}


\section{System Architecture}
In order to accomplish the requirements, it is important to consider design choices that could contribute towards qualities that are more adequate for this kind of development. For this, a trade-off between the three qualities of the software will be considered as shown in Figure \ref{fig:balance_attributes}, as it is not always possible to have everything in a system due to the possible conflict of attributes. For example, a highly animated and attractive design would affect the performance of the system and its capability in accomplishing the primary goal. Architectural choices will be taken into account over trade-off.


\begin{figure}[H]
\begin{adjustbox}{width=.5\textwidth,center=\textwidth}
  \centering
  \includegraphics[scale=1]{images/balanceCircles.png}
\end{adjustbox}
  \caption[Finding a balance between software attributes]{Finding a balance between software attributes.}
  \label{fig:balance_attributes}
\end{figure}

\subsection{Development Approach and Operating System}
It is possible that the development could take a hybrid or a native approach, as summarised in Table \ref{tab:development_approaches}. In a nutshell, it means that the software could be developed in a web-browser container and therefore be compatible with all devices able to display modern web-pages. Or it could even be be developed using the native software development kit for each system (Android, iOS, etc) therefore making it compatible with that specific operating system. A native development approach carries with it the disadvantage that a new source code would be needed for each operating system. Alternatively it would bring many benefits such as the response speed, being able to use advanced graphics and animation techniques and having an easy access to the native APIs and components of the device (e.g. GPS and gyroscope). In order to gain performance and usability a native approach has been chosen. Consequently, it is important to define the operating system target, which will solely be based on the market share of the most important operating systems according to IDC. \footnote{\url{http://www.idc.com/prodserv/smartphone-os-market-share.jsp}} Android holds a commanding share of 82.8\%, compared to 13.9\% of iOS, and 2.6\% for the Windows phone. Because of this, to target a wider range of mobile phone users the selected operating system for developing the application is Android.

\begin{table}[ht]
\centering
\begin{adjustbox}{width=.6\textwidth,center=\textwidth}
\begin{tabular}{lrr}
  \hline
   - & Native & Hybrid  \\ \hline
   Language & Switf or Java & HTML and Javascript \\
   Speed & Fast & Medium \\
   Portability & None & High \\
   Advanced Graphics & High & Moderate \\
   Access to native APIs & High & Moderate \\
   Development cost & Expensive & Reasonable \\
   \hline
\end{tabular}
\end{adjustbox}
  \caption[Native vs hybrid development approach ]{Native vs hybrid development approach (Adapted). \footnotemark }
\label{tab:development_approaches}
\end{table} 
\footnotetext{\url{http://julyrapid.com/hybrid-vs-native-mobile-app-decide-5-minutes/}}

\subsection{Air Quality Data Source}
It was considered to acquire personal air quality sensors to provide real-time high-resolution data to the system; unfortunately, such devices are still not very feasible as they are costly and become obsolete over time due to loss of accuracy. One example is the Libelium Gases PRO sensor, which costs around \euro{}2000 and has an approximate lifetime of 12 months.\footnote{\url{http://www.libelium.com/calibrated-air-quality-gas-dust-particle-matter-pm10-smart-cities/}} 

As mentioned in the background, there are currently many automatic fixed sensors around Scotland that include a variety of pollutants and provide real-time data. However, the data of these devices is not exposed through any public API from the Scottish government but as plain HTML. One possibility as well is using any third party air quality data source such as the Breezometer air quality API\footnote{\url{https://breezometer.com/air-quality-api/overview/}} or the OpenAQ API\footnote{\url{https://openaq.org/#/}}. The first one is a private source of air quality data which bills for the queries to the service and thus not very cost-effective. The second is a public API but the number of sensors from each city in Scotland are limited compared to the official government sources (the Air Quality in Scotland website \cite{Scottishairquality.co.uk2016}). An example of this is the city of Glasgow, which from the official sources are 26 fixed automatic sensors but from the public API only 4 are available. As a result, to have a more accurate air quality source the data will be collected from the official sources and exposed to later query from the device. 


\subsection{Infrastructure As a Service}

IaaS or Infrastructure as a Service is a cloud service that provides computing infrastructure on demand. The benefits of using such a service as opposed to traditional approaches are many. First, there is no need to buy, install and maintain a server at a fixed location, reducing associated costs. Second, setting up an IaaS can be done within minutes, saving  development time. The following options were considered for setting up an IaaS: 

\begin{itemize}
	\item Google Cloud Platform\footnote{\url{https://cloud.google.com/}}
    \item Amazon Web Services\footnote{\url{https://aws.amazon.com/}}
    \item Microsoft Azure\footnote{\url{https://azure.microsoft.com/en-us/}}
\end{itemize}

Due to budget constraints, the most cost-effective solution was chosen. Amazon's free tier allows up to 12 months of free usage, as opposed to Microsoft's 30 days trial and Google's 60 days trial.

\subsection{Back-end technological choices}
The back-end component of the system will perform two functions: 

\begin{itemize}
	\item Gather the required air quality data and insert it into a database. The easiest way to do this is via any available web-crawling libraries, such as Jspider\footnote{\url{http://j-spider.sourceforge.net/}} for Java or Scrapy \footnote{\url{http://scrapy.org/}} for Python. Scrapy was chosen because it is a complete and stable framework that can handle crawling operations in an easy way. 
	\item Provide service to the mobile application. There is a need for a logic tier to process the input elements for the health advice. As Java is a stable and reliable framework,  it is well suited for most of the tasks that require server side processing and client handling in an optimal way.
\end{itemize}

\subsection{Schema-free Database}

Air quality readings will need to be stored for later query from the device. For this, there is the option of using standard SQL databases or schema-free databases. Analysing the data that will be stored in further detail, it will contain the readings gathered from each sensor in Scotland, and specific information such as their name, location and type of sensor. The data would be extracted in the form of JSON documents without normal relations between them. Moreover, the transactions between the devices and the database should be as fast as possible without immediately considering instant reliability as sensors are updated at best each hour. Another advantage of these types of databases is the agility they provide by eliminating the need of schemas. Taking into account these factors, it is more suitable to use a schema-free database or NoSQL. Specifically, it will be used a Dynamo NoSQL from the same Amazon Web Services infrastructure because it integrates easily with the already chosen IaaS.

\subsection{Overall Design}
The overall design of the infrastructure is as shown in Figure \ref{fig:architecture}. The back-end and the database components run on Amazon Web Services (AWS). The Android device uses the AWS Software Development Kit (SDK) to make queries to the Dynamo database directly to read the air quality data, this to facilitate the mapping between the objects in the database and the Android device. To access the advice service, the approach is to make HTTP GET requests to the Java containers hosted in AWS. 
\begin{figure}[H]
\begin{adjustbox}{width=.65\textwidth,center=\textwidth}
  \centering
  \includegraphics[scale=1]{images/architecture.png}
\end{adjustbox}
  \caption[Architecture design]{Architecture design.}
  \label{fig:architecture}
\end{figure}

\section{Mobile Application}
The Android application will be the visible interface for the final user. At this point it is key to understand the design choices that can affect our main quality attributes. These are the development of the user interface and the lower-level issues related to the management of the screen components.

\subsection{User Interface}
Three distinct screens will achieve the functional requirements as shown in Figure \ref{fig:chain_of_screens}. The first screen will be responsible for displaying a general overview and personalised advice. The second screen will show individual pollutants and the third individual pollutants over time. There will be another initial screen to allow users to input the personal details needed for the application.

One useful guideline at the design level to connect the screens with each other is simplicity. Having many different screens for different purposes could overwhelm the user, so they will be accessible at every time using a tabbed menu and enabling gestures to navigate through them. That is, the screens will become active by selecting a tab, or by swiping between them.

\begin{figure}[H]
\begin{adjustbox}{width=.45\textwidth,center=\textwidth}
  \centering
  \includegraphics[scale=1]{images/screenChain.png}
\end{adjustbox}
  \caption[Chain of screens]{Chain of screens.}
  \label{fig:chain_of_screens}
\end{figure}

\subsection{Activities}
The previous screens will be developed individually by the use of activities. An Activity is the abstraction employed by the Android framework to contain the visual and run-time elements of the currently displayed screen. The way Activities are created and maintained in the screen is quite unusual in contrast with standard Java Swing interfaces. It makes use of an event driven life-cycle that must be understood beforehand for a smooth and error-free behaviour. 

As available RAM is limited on a handheld device and it has to be shared with all other running applications, it is not possible to maintain all Activity instances in memory. The Activity lifecycle is a workaround for this problem, allowing the Dalvik virtual machine (A virtual machine that runs all Android programs) to determine automatically which activities will be kept in memory. Thus, the activities should provide behaviour to be called from any point in the lifecycle.

The lifecycle is illustrated in Figure \ref{fig:activities_lifecycle}. It can be observed that the first method called by Android is the \textit{onCreate()} method, used to instantiate the requested Activity and any screen components. Later, the application can pass to other states until it is stopped and started again from a different entry point to the one it was previously created from. The activities should be able to handle both starting points and restore any state left by the user when the Activity was stopped. 

\begin{figure}[H]
\begin{adjustbox}{width=1\textwidth,center=\textwidth}
  \centering
  \includegraphics[scale=1]{images/basic-lifecycle.png}
\end{adjustbox}
  \caption[Android activities lifecycle]{Android activities lifecycle.\footnotemark}
  \label{fig:activities_lifecycle}
\end{figure}
\footnotetext{\url{https://developer.android.com/training/basics/activity-lifecycle/starting.html}}

\subsection{Layouts}
The main component to render a screen in Android is the Layout. Layouts make use of XML to define the objects on the screen. As shown in the example below, it renders a \textit{TextView} object followed by a \textit{Button} object. Layouts need to be attached to an Activity or Fragment to give them behaviour. Once attached, the components defined by the Layout can be extracted by using their ID's and the method \textit{findViewByID}
\begin{verbatim}

<?xml version="1.0" encoding="utf-8"?>
<LinearLayout xmlns:android="http://schemas.android.com/apk/res/android"
              android:layout_width="match_parent"
              android:layout_height="match_parent"
              android:orientation="vertical" >
    <TextView android:id="@+id/text"
              android:layout_width="wrap_content"
              android:layout_height="wrap_content"
              android:text="Text" />
    <Button android:id="@+id/button"
            android:layout_width="wrap_content"
            android:layout_height="wrap_content"
            android:text="Text" />
</LinearLayout>
\end{verbatim}

It is important to understand the correct coupling of Android layouts because this influences how fast the screen is rendered. The rendering process is called \textit{inflating} and occurs in the \textit{onCreate()} method. As the Activities are rendered on the screen, children components are attached and may call to render their own layout, potentially slowing down the the application. For this purpose, the Android SDK makes available the Android device monitor to understand how much time is being taken to render all the components displayed at a given time. 

The Android device monitor helps to visualise the components taking part on the screen at some point in run-time. It shows  the topmost (the root) to the bottommost element (the leaf), how they are attached and which element of the entire tree is taking the most time to render, therefore helping in the debugging process. As shown in Figure \ref{fig:android_device_monitor}, the delaying components are marked with red dots which means that the rendering of that specific screen component is among the slowest half of views. The yellow means the view renders faster than the bottom half of the other views and the green means that renders faster than at least half of the other views. Understanding this tool helps achieving the required performance quality attribute. 

\begin{figure}[H]
\begin{adjustbox}{width=1\textwidth,center=\textwidth}
  \centering
  \includegraphics[scale=1]{images/android_device_monitor_2.png}
\end{adjustbox}
  \caption[Android device monitor]{Android device monitor.}
  \label{fig:android_device_monitor}
\end{figure}

\subsection{Material Design}
As well as understanding how to achieve a smooth behaviour within the application it is wise to revise the good use of design principles available to enhance usability and its implied attributes, ease of use and attractiveness.

According to Google: \begin{displayquote}Material design is a visual language for our users that synthesises the classic principles of good design with the innovation and possibility of technology and science. \end{displayquote} 

The material design framework helps with principles that address design issues to allow every application user to navigate, understand and use the designed user interface (UI) successfully. It also serves to unify the experience across different devices and screen sizes, so that users will be able to understand naturally design cues of new applications. The following principles are adhered to: 

\begin{itemize}
    \item Material: The material metaphor is based on paper and ink like in the real world, in three dimensions using lights and shadows. The intention is to close the gap between the perception of real world elements and digital elements. 
    \item Graphics: The use of graphical elements such as typographies, icons, adequate colours and images make a pleasant interface and give hierarchy and meaning. 
    \item Motion: User actions that initiate motion and transform the state of the interface serve as playful cues that help to understand the flow of the interface as well as to provide continuity and feedback. 
\end{itemize}


\begin{figure}[H]
\begin{adjustbox}{width=1\textwidth,center=\textwidth}
  \centering
  \includegraphics[scale=1]{images/material_google.png}
\end{adjustbox}
  \caption[Android material design framework]{Android material design framework.\footnotemark}
  \label{fig:android_material_design}
\end{figure}
\footnotetext{\url{https://material.google.com/}}

\subsection{Targeted Devices}
As the Android OS and APIs (Application Programming Interfaces) are updated over time some applications built with later SDK versions will not be compatible with earlier OS versions. Thus, it is important to define the earliest operating system that will be included in the development. For this, considerations such as the number of users that are still using old versions of Android and new features that might be useful from newer SDK versions should be studied. For instance, the most used OS distribution nowadays is KitKat which makes use of an API level 19 as shown in Figure \ref{fig:android_platform_versions}.  Unfortunately, it does not offer native support for material elements, which are a strong positive for this development. A workaround for this problem is to include the Android support library\footnote{\url{https://developer.android.com/topic/libraries/support-library/index.html}} which will provide compatibility for features made available in newer versions of Android. The final targeted API is level 19 (KitKat) for the use of an up to date stable API and to include up to the 79\% of the Android users.  

\begin{figure}[H]
\begin{adjustbox}{width=1\textwidth,center=\textwidth}
  \centering
  \includegraphics[scale=1]{images/android_platform_versions.png}
\end{adjustbox}
  \caption[Devices running a given version of Android]{Devices running a given version of Android.\footnotemark}
  \label{fig:android_platform_versions}
\end{figure}
\footnotetext{\url{https://developer.android.com/about/dashboards/index.html}}

\section{Back-end}
Considering in detail, the back-end architecture will be composed as shown in Figure \ref{fig:architecture_back_end_detail}. A virtual machine will run Ubuntu Linux because it is a very stable and compatible version of Linux. There was also a possibility to use a custom Amazon flavour of Linux, which was not viable due software compatibility issues. The back-end will host two main components to power the application, the data service and the advice service. 

\begin{figure}[H]
\begin{adjustbox}{width=.6\textwidth,center=\textwidth}
  \centering
  \includegraphics[scale=1]{images/architecture_back_end_detail.png}
\end{adjustbox}
  \caption[Back-end architecture]{Back-end architecture detail.}
  \label{fig:architecture_back_end_detail}
\end{figure}


\subsection{Data Service}
 The air data source is the air quality in Scotland website. As it displays the updated readings in HTML, a web crawler is needed to navigate the readings making use of XPATH selectors to extract the required data. Once extracted, it will be inserted in JSON format directly to the Dynamo database.
 
A daemon is a program that runs in the background autonomously and will call the web crawler every half hour to get the most up-to-date readings from the air quality readings (They are updated at the most each hour). 

\subsection{Advice Service}
The advice service will be responsible for providing health advice based on the COMEAP report \cite{HealthProtectionAgencyfortheCommitteeontheMedicalEffectsofAirPollutants2011}. Before analysing the techniques that can be considered to translate the advice from the COMEAP report into a service, there are few matters to notice. The document has different categories of advice based on the sensitivity and age of the people. They are not simple data statements that can be easily used from a database, but more of logical statements. For example, from Figure 2.3 we can state that at-risk individuals, who are adults or children when the AQ index is from 7 to 9 points should reduce strenuous physical exertion. Also, it is likely that the health statements would need to be modified or expanded in the future. 

When it comes to providing logic or \textit{intelligence} to a system for it to make choices, such as the health advice, there are two general approaches that may be considered: rule-based systems and machine learning techniques. Rule-based systems or expert systems are a good fit when all the options for the system are known beforehand, and when all the conditions can be easily written as \textit{if else statements}. On the other hand machine learning techniques are useful when the datasets are fairly large and the rules have to be made at execution time. Therefore, the easiest way to provide the system with the capability to give the required health advice is via an expert system.

\section{First Prototype}
Following the sketched application design and the interaction design methodology, the first prototype was drawn in order to meet the user requirements and to start defining how the final interface would look like. As mentioned, the application would be composed of three final distinct visualisations for the main functional requirements:

\begin{itemize}
    \item Visualise current air quality and provide health advice.
    \item Visualise individual pollutants status and information about their sources and effects.
    \item Visualise individual pollutants over time.
\end{itemize}


\begin{figure}[H]
\begin{adjustbox}{width=1.2\textwidth,center=\textwidth}
  \centering
  \includegraphics[scale=1]{images/firstPrototype.png}
\end{adjustbox}
  \caption[Frist prototype]{First prototype. From left to right: first, second and third visualisation.}
  \label{fig:first_visualization_first_prototype}
\end{figure}


\subsection{Screen 1}

The first visualisation would contain first-hand pertinent information that would adhere to different functions. It would convey the current air quality status and its components, provide information about the source of the air quality reading, show air quality advice given the user configuration and finally let the user configure the air quality advice in a playful way. We could think of this screen as a dashboard that will give the user the required basic information to use the rest of the application. In fact, the display of this screen is enough to get a full picture of what is going on without much detail.
This screen is shown in Figure \ref{fig:first_visualization_first_prototype}. It is visually divided into the top and bottom component. The top component is information about the sensor so as to give the user confidence about the reading by means of a map, the hour it was last updated, and the name location of the sensor. The bottom component contains two bars: one for adjusting the sensitivity level, and the other to show the air quality status followed by the health advice in text and the pollutants at play in the reading at the point it was taken. 

\subsection{Screen 2}

The second visualisation aims to provide an individual visual hint for each pollutant. The need for this visualisation arises from the difficulty of to observe how the readings are at a pollutant level. The overall idea is to show a list of the pollutants that are present at the moment with a visual cue to allow for an immediate insight of the reading as well as containing the official measure and the measurement unit. Also, by clicking in each pollutant, it will be possible to examine more specific information about the pollutant.

\subsection{Screen 3}

The final visualisation will provide full detail of how the readings were behaving throughout the day, or on a particular date. The need for this visualisation arose from the need for sensitive users to relate their symptoms to pollution spikes and to gain knowledge about what might be affecting them at a pollutant level and support decision making. For this purpose, a line graph is employed as is an adequate tool to show readings over time. Also, the same pollutants that were taking part in the first and second visualisations will be consistent in this screen. The difference is that they will be touch-enabled to allow the user to select which pollutant to visualise as a way of providing interaction. At the top of the screen there will be controls to define the specific date and time of the readings.

\chapter{Design}
\section{Design goals}
\subsection{Trade-off}
\input{5-design/2-Front-End-Design}
\input{5-design/3-Back-End-Design}
\chapter{Implementation}
\section{Back end implementation}
\subsection{Database}
\subsection{Web crawler}
\subsection{Advice service}
\section{Front end implementation}
\subsection{User interface}
\subsection{Visualizations}
\subsection{Demo mode}
\input{6-implementation/3-Deployment}
\chapter{Evaluation}
As mentioned in the methodology, the evaluation was carried out during all the development lifecycle allowing improvements with respect to each iteration. This chapter describes how the evaluation of a final prototype was carried out to assess functional and non functional aspects, as well as the technical evaluations that were performed during the development to ensure the accomplishment of the software quality attributes. 

\section{Technical Evaluation}

\subsection{Automated tests}

Testing an Android application is particularly challenging because as opposed with iOS, many manufacturers are allowed to distribute their own devices that run this operative system. This means that an application should be able to work in many different phones with different screen sizes, resolutions or specific hardware components. There is not a proven way to test an android application other than trying it out on many distinct devices. This normally is an expensive task, but nowadays emerging services make available cloud based devices so that developers can connect to them remotely to install and test their applications on demand.

Amazon provides such service in the AWS mobile device farm \footnote{\url{https://aws.amazon.com/device-farm/}}. The application was tested by executing automated tests that explore the interface autonomously to discover how it might react to normal user interaction but also under odd circumstances. Another benefit of this service is that the tests are executed in parallel with a number of devices that are widely used (in this case 15) to encounter errors that might arise from different screen types or hardware components. 

The test starts by uploading a packaged version of the application to the service who is responsible to install it in different devices of the farm. If the application is able to respond to all simulated user input and keep working under a defined period of time then the test for that specific device is passed. Two iterations of this test allowed to find bugs and interface errors that were corrected and optimized. In the first test carried out the application was failing on 5 from 15 devices, the logs of the crash were studied and fixed. A second iteration showed a better outcome, just 1 of the 15 devices were failing the test. Resulting in an improved compatibility.   

\begin{figure}[H]
\begin{adjustbox}{width=1\textwidth,center=\textwidth}
  \centering
  \includegraphics[scale=1]{images/automated_tests.png}
\end{adjustbox}
  \caption[Automated tests on AWS]{Automated tests on AWS}
  \label{fig:automated_tests}
\end{figure}

\subsection{Resource usage}
In order to ensure that the application consumes the adequate number of resources from any device, it is straightforward to inspect the resource usage of the running application. Android provides tools to monitor the usage of the consumed resources such as battery, bandwidth and memory from any application installed. From its usage the following was identified:
\begin{itemize}
	\item Memory: The application consumes in total 33mb in storage, which is not extraordinary compact but is feasible given the modern phones storage capabilities. 
    \item Data usage: For all the development process the application was executed several times during almost a month. It used 2.15mb of data in total. Which is a very good performance for an application of this kind.
    \item Battery usage: From using the application on repeated occasions in a dedicated device, it was found that the application doesn't reach the list of top battery consumer services in the phone. The reason for this is that the it doesn't employ any background process, all the communication tasks are performed when the app is open. Also, it does not require the GPS to be on as it will use the location already gathered by the internal phone services.
\end{itemize}


\begin{figure}[H]
\begin{adjustbox}{width=.45\textwidth,center=\textwidth}
  \centering
  \includegraphics[scale=1]{images/resource_usage.png}
\end{adjustbox}
  \caption[Application resource usage]{Application resource usage}
  \label{fig:automated_tests}
\end{figure}


\section{Evaluation with users}

The evaluation of a third prototype was carried out with potential final users of the application. That is, with volunteers from the AUKAR group, and from general users from the University of Edinburgh. Two methods were employed for this purpose, a usability test complemented by a design critique. 

\subsection{Usability testing}
To ensure that the functional requirements as well as the usability attributes are accomplished, it is helpful to test the product with real users under real world circumstances. This would outcome on getting a more accurate feedback on how well users are interacting with the product in terms of the remaining selected attributes: usability and performance.

In total there were three participants, one from AUKAR, and two from the University of Edinburgh. All of them were new to the application. For the set-up, an android phone (Motorola Moto G 3rd Gen.) was configured with the AirQueue app and a screen logger. The room was equipped with a video recorder to keep log of the said by the participants. The test was divided in two parts in a one hour individual meeting. 

The first part of the test consisted in executing small tasks or scenarios that should be accomplished within the interface, covering tasks in all visualizations and screens. Participants were asked to perform them individually and to indicate if they were able to do so with ease. Also, some of them were thinking aloud during the test. 

P1, P2 and P3 are the participants of the test. Their ages were 68, 28 and 27 respectively. Their occupations are retired, designer and biologist. Each participant marked whether they were able to finish each of the tasks with ease as instructed with n (No) or y (yes). The findings and the scenarios are shown in Table \ref{tab:test_scenarios}.

In general the users were able to follow up the instructions one by one using the visual and textual cues provided. From the screen video logging was found that users were trying to interact with the elements of the screen to discover their usage (even if some components were not providing any usage), also, the tool-tips served a good guiding function, all the participants used them to understand better the navigation through the interface. The screen with the better success rate was (PEND). 

\begin{figure}[H]
\begin{adjustbox}{width=.8\textwidth,center=\textwidth}
  \centering
  \includegraphics[scale=1]{images/scenarios_rate.png}
\end{adjustbox}
  \caption[Scenario tests completion rate]{Scenario tests completion rate}
  \label{fig:automated_tests}
\end{figure}

\newcommand{\specialcell}[2][c]{%
  \begin{tabular}[#1]{@{}l@{}}#2\end{tabular}}

\begin{table}[H]
\centering
\begin{adjustbox}{width=1\textwidth,center=\textwidth}
\begin{tabular}{llrrr}
  \hline
   Tab/Screen & Activity & P1 & P2 & P3 \\ \hline
   Overview & \specialcell[t]{1.-I want to start the application and reach the first\\'overview' screen.}  & y & y & y \\
   Overview & \specialcell[t]{2.-I want to visualize the location of the closest air\\quality-sensor and know when was it last updated. (Just\\read the information).} & y & y & y \\
   Overview & \specialcell[t]{3.-I want to adjust my sensitivity level to indicate I have\\high sensitivity.} & y & y & y \\
   Overview & \specialcell[t]{4.-I want to know the air quality index. (How good or bad\\the air quality is).} & y & y & y \\
   Overview & \specialcell[t]{5.-I want to read my personalized health advice. (Just\\read the information)} & y & y & y \\
   Overview &\specialcell[t]{6.-I want to adjust again my sensitivity level to indicate I\\have a low sensitivity and read my advice again.} & y & y & y \\
   Pollutants &\specialcell[t]{7.-I want to navigate to the second 'Pollutants' screen.} & y & y & y \\
   Pollutants &\specialcell[t]{8.-I want to examine the measured value for the sulphur\\dioxide pollutant.} & n & y & y \\
   Pollutants &\specialcell[t]{9.-I want to know if the measured value for sulphur\\dioxide is categorized as good, regular, bad or extremely\\bad.} & y & y & y \\
   Pollutants &\specialcell[t]{10.-I want to know further information about particulate\\matter. I want to read the sources and health effects of\\
   this specific pollutant.} & y & y & y \\
   Graphs &\specialcell[t]{11.-I want to navigate to the third 'Graphs' screen.} & y & y & y \\
   Graphs &\specialcell[t]{13.-I want to select the 'CO' pollutant and visualize the\\measured values through the day.} & y & y & y \\
   Graphs &\specialcell[t]{14.-I want to select the PM10 pollutant and visualize the\\measured values for yesterday.} & y & y & y \\
   \hline
\end{tabular}
\end{adjustbox}
  \caption[Usability testing scenarios]{Usability testing scenarios and results}
\label{tab:test_scenarios}
\end{table} 

The second part of the usability test was guided by an usability scale. The users were asked to evaluate their experience ranking their agreement over different statements. Where 1 was strongly disagree and 5 was strongly agree. The employed usability scale and the responses are shown in Table \ref{tab:test_usability_scale}.

\begin{table}[H]
\centering
\begin{adjustbox}{width=1.2\textwidth,center=\textwidth}
\begin{tabular}{llrrr}
  \hline
   - & Statement & P1 & P2 & P3 \\ \hline
   Overview & \specialcell[t]{1.-I think that the first screen (overview) was easy to\\understand/navigate.} & 3 & 1 & - \\
   Overview &\specialcell[t]{2.- I think that the 'overview' screen would help me to understand the\\current situation of air quality.} & 5 & - & - \\
   Overview &\specialcell[t]{3.- I think that the personalized health 'advice' would help me to take better\\choices through the day.} & 5 & - & - \\
   Overview &\specialcell[t]{4.- I feel that is useful to know the location of the closest air quality sensor.} & 5 & - & - \\
   Overview &\specialcell[t]{5.- I think that the colours of the 'overview' screen help me to\\understand the usage of the screen components.} & 1 & - & - \\
   Overview &\specialcell[t]{6.- In the 'overview' screen, I think that the wording of the menus, labels and\\pop-ups is clear and concise.} & 3 & - & - \\
   Pollutants &\specialcell[t]{7.- I think that the second screen (pollutants) was easy to\\understand/navigate.} & 5 & - & - \\
   Pollutants &\specialcell[t]{8.- The 'pollutants' screen would help me to understand the current situation of air\\quality.} & 5 & - & - \\   
   Pollutants &\specialcell[t]{9.- In the 'pollutants' screen the individual pollutant circles help me to understand\\easily how good or bad the measurements are. } & 5 & - & - \\   
   Pollutants &\specialcell[t]{10.- It is useful to have information about the sources and effects of air pollution. } & 5 & - & - \\   
   Graphs &\specialcell[t]{11.- I think that the third screen (graphs) was easy to understand/navigate.} & 5 & - & - \\   
   Graphs &\specialcell[t]{12.- I think that the third screen 'Graphs' would be useful to track my response to\\certain pollutants.} & 5 & - & - \\   
   Graphs &\specialcell[t]{13.- In the graphs screen the pollutant different colours help me to differentiate them.} & 5 & - & - \\   
   Graphs &\specialcell[t]{14.- In the third screen (Graphs) the pollutant graphs are displayed promptly.} & 5 & - & - \\   
   App in general &\specialcell[t]{15.- I think that it was easy to access and navigate through all the three screens.} & 4 & - & - \\     
   App in general &\specialcell[t]{16.- I would imagine that most people would learn to use this system very quickly.} & 4 & - & - \\ 
   App in general &\specialcell[t]{17.- I needed to learn a lot of things before I could get going with this system.} & 5 & - & - \\ 
   App in general &\specialcell[t]{18.- I would say that the colours of the application help me to understand it better.} & 5 & - & - \\ 
   App in general &\specialcell[t]{19.- I think that the application response to my actions wast fast and smooth.} & 5 & - & - \\    
   App in general &\specialcell[t]{20.- The application starts promptly } & 5 & - & - \\       
   App in general &\specialcell[t]{21.- I thought that simple indicators such as 'good', 'regular', 'bad' helped to understand\\the current air quality status.} & 5 & - & - \\       
   App in general &\specialcell[t]{22.- I thought that colour indicators (green/yellow/red) helped to understand the\\current air quality status.} & 5 & - & - \\          
   App in general &\specialcell[t]{23.- I would use this application to make better choices about my health.} & 5 & - & - \\             
   App in general &\specialcell[t]{24.- I would use this application to know more about pollution in general.} & 5 & - & - \\                
   App in general &\specialcell[t]{25.- I think using this application is fun and enjoyable.} & 3 & - & - \\         
   App in general &\specialcell[t]{26.- Having an 'smart' health advice   would help making my life easier.} & 5 & - & - \\      
   App in general &\specialcell[t]{27.- It is more engaging or interesting using an application instead of a website to get\\information about pollution} & 5 & - & - \\         
   \hline
\end{tabular}
\end{adjustbox}
  \caption[Usability testing scenarios]{Usability testing scenarios and results}
\label{tab:test_usability_scale}
\end{table} 

\subsection{Design critique}
At the end of the meeting, participants were asked to give their thoughts on the application. This was an open discussion where the participants where guided through a couple of questions to find out more about what did they like, dislike, find useful or would add to the application. 

\bigskip
\textbf{What do you think about the interface?}
\bigskip

\begin{itemize}
	\item "The interface is very clean."
    \item "It looks nice."
    \item "This is almost like the front page on a book isn't it?" - Referring to the first screen.
    \item "I didn't realize how to move away this tool-tip."
\end{itemize}

\bigskip
\textbf{Do you think having an application like this would help you to stay more informed about air pollution?}
\bigskip

\begin{itemize}
	\item "It would, because that way I can decide what to do based on what I see."
	\item "Certainly It would be a great help."
\end{itemize}

\bigskip
\textbf{If you could add something, what would it be?}
\bigskip

\begin{itemize}
	\item "I was expecting this to react, it didn't occur to me that it was an introduction."
    \item "Maybe the ability to select other sensors from this screen."
\end{itemize}

\bigskip
\textbf{What do you like about the application?}
\bigskip

\begin{itemize}
	\item "I like the way you can adjust, presumably once you've adjust it, everything else in the screen takes it into account."
\end{itemize}


\chapter{Conclusions and Future Work}
Vi


\bibliography{mendeley}
\bibliographystyle{IEEEtranN}
\appendix
\chapter{Requirements Elicitation Survey} 
\begin{figure}[H]
\begin{adjustbox}{width=.8\textwidth,center=\textwidth}
  \centering
  \includegraphics[scale=1]{surveys/q1.png}
\end{adjustbox}
\end{figure}
\begin{figure}[H]
\begin{adjustbox}{width=1\textwidth,center=\textwidth}
  \centering
  \includegraphics[scale=1]{surveys/q2.png}
\end{adjustbox}
\end{figure}
\begin{figure}[H]
\begin{adjustbox}{width=1\textwidth,center=\textwidth}
  \centering
  \includegraphics[scale=1]{surveys/q3.png}
\end{adjustbox}
\end{figure}
\begin{figure}[H]
\begin{adjustbox}{width=1\textwidth,center=\textwidth}
  \centering
  \includegraphics[scale=1]{surveys/q4.png}
\end{adjustbox}
\end{figure}
\begin{figure}[H]
\begin{adjustbox}{width=1\textwidth,center=\textwidth}
  \centering
  \includegraphics[scale=1]{surveys/q5.png}
\end{adjustbox}
\end{figure}
\begin{figure}[H]
\begin{adjustbox}{width=1\textwidth,center=\textwidth}
  \centering
  \includegraphics[scale=1]{surveys/q6.png}
\end{adjustbox}
\end{figure}


\end{document}
